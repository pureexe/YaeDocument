\documentclass[hidelinks,a4paper,14pt]{article}
\usepackage{ amsmath, amssymb}
\usepackage{graphicx}
\usepackage{color,amsmath,graphics,graphicx}
\usepackage{amsfonts}
\usepackage{mathrsfs,hyperref}
\usepackage{latexsym,amsmath,enumerate,amsbsy,amsthm}
\textwidth = 415pt

%==============================================
\usepackage{fontspec}
\usepackage{xunicode}
\usepackage{xltxtra}
\defaultfontfeatures{Scale=1.23}
\XeTeXlinebreaklocale “th_TH” % สำหรับตัดคำ
\setmainfont[Scale=1.23]{TH SarabunPSK}
%==============================================
%%%%%%%%%%%%%%% THEOREM Environments %%%%%%%%%% 
\newtheorem{theorem}{ทฤษฎีบท}[section]						%
\newtheorem{lemma}[theorem]{บทตั้ง}						%
\newtheorem{conjecture}[theorem]{บทคาดการณ์}				%
\newtheorem{definition}[theorem]{บทนิยาม}					%
\newtheorem{remark}[theorem]{หมายเหตุ}						%
\newtheorem{proposition}[theorem]{ประพจน์}					%
\newtheorem{corollary}[theorem]{บทแทรก}					%
\numberwithin{equation}{section}							%
\newtheorem{example}[theorem]{ตัวอย่าง}						%
%\newtheorem{exercise}{แบบฝึกหัด}[chapter]	
%\renewcommand{\chaptername}{บทที่}
\renewcommand\tablename{ตารางที่}
\renewcommand\figurename{รูปที่}
\renewcommand{\contentsname}{สารบัญ}						%
%\renewcommand{\bibname}{บรรณานุกรม}						% 
\renewcommand{\indexname}{ดรรชนี}					%
%%%%%%%%%%%%%%%%%%%%%%%%%%%%%%%%%%%%%%%%%%%%%%%
% addition mod
\usepackage{subcaption,float,framed,algorithm2e,hyperref}
%%%%


\begin{document}
	{\begin{center}
			\textbf{Project Proposal}\\
			\vspace{0.5cm}
			\textbf{Division of Applied Mathematics, Department of Mathematics}\\
			\vspace{0.5cm}
			\textbf{Faculty of Science, Silpakorn University}
		\end{center}}
		{
			\vspace{0.5cm}
			\flushleft{Date : { 27 กันยายน 2561 } \hfill{ }}
			\flushleft{Advisor : {  ผู้ช่วยศาสตราจารย์ ดร. นพดล  ชุมชอบ}}\\
			\flushleft{Student : { นายภัคพล       พงษ์ทวี     รหัส  07580028}
			\vspace{1cm}
		}
		
		
		% Here the project title
		{\textbf{\begin{flushleft}Project Title : ขั้นตอนวิธีเชิงตัวเลขชนิดใหม่สำหรับการต่อเติมภาพที่ใช้การแปรผันรวมกับการประยุกต์สำหรับซ่อมแซมภาพวาดศิลปะไทยและการลบบทบรรยายจากอนิเมะ \\
				(A new numerical algorithm for TV-based image inpainting with its applications in restoring Thai painting images and removing subtitles from animes)
			\end{flushleft}
		}}
		\thispagestyle{empty}
		
	\section{ที่มาและความสำคัญ}
	
		\hspace{1cm} ภาพดิจิตัล (digital images)  คือภาพที่นิยมใช้กันอย่างแพร่หลายในปัจจุบันอาจจะสร้างได้หลายวิธีทั้งการใช้กล้องถ่ายภาพเพื่อให้ได้ภาพ หรืออาจจะใช้อุปกรณ์ทางการแพทย์ต่างๆ จนไปถึงการใช้คลื่นที่มองไม่เห็นเพื่อถ่ายภาพดาราจักรต่างๆ ในอวกาศ  ซึ่งภาพที่ได้ออกมานั้นมักจะผ่านการประมวลการประมวลผลอยู่เสมอ ตัวอย่างเช่น ภาพถ่ายพื้นผิวดวงจันทร์เมื่อส่งสัญญาณกลับมาจากดาวเทียมจะมีสัญญาณรบกวนเข้ามาแทรก จึงจำเป็นที่จะต้องผ่านการกำจัดสัญญาณรบกวนออกจากภาพ (images denoising) การติดตามอาการคนไข้ที่มีอาการเนื้องอกจะเป็นต้องทำการลงทะเบียนภาพ (image registration) เพื่อให้แพทย์สามารถติดตามการเปลี่ยนแปลงของเนื้องอกได้ การติดตามรถที่กระทำผิดกฏหมายจราจร จำเป็นต้องแยกรถยนตร์ออกจากพื้นหลังโดยใช้การแบ่งส่วนภาพ (image segmentation) และการลบวัตถุที่ไม่ต้องการออกไปจากภาพจะใช้การต่อเติมภาพ (image inpainting) เป็นต้น 
		
		 \hspace{1cm}  การต่อเติมภาพ เป็นหนึ่งในกระบวนการประมวลผลภาพที่จะเติมเต็มข้อมูลที่หายไปในพื้นที่ภาพที่กำหนด โดยมีจุดประสงค์เพื่อซ่อมแซมภาพที่เสียหาย โดยพื้นที่ภาพส่วนนั้นไม่สามารถพบได้จากการสังเกต โดยการกู้คืน สี, โครงสร้าง และพื้นผิว ที่เกิดการเสียหายเป็นวงกว้าง พิกเซลที่จะนำมาใช้ซ่อมแซมจะถูกคำนวณขึ้นมาใหม่จากข้อมูลที่พิกเซลที่อยู่โดยรอบที่ยังไม่เสียหาย \cite{ref:defination-of-inpaint}  ซึ่งใช้ลบสิ่งที่ไม่ต้องการออกจากภาพ ปัจจุบันมักเห็นได้ตามแอปพลิเคชันหน้าใส ที่ช่วยลบริ้วรอยที่ไม่ต้องการออกจากใบหน้า
		 
		 \subsection{การประยุกต์ใช้สำหรับซ่อมแซมภาพวาดศิลปะไทย}
		 
		 \hspace{1cm}เนื่องจากการต่อเติมภาพด้วยการแปรผันจะสนใจที่ความต่อเนื่องของโครงสร้างทางเรขาคณิต ซึ่งวิธีการต่อเติมภาพรูปภาพด้วยการแปรผันมักจะให้ผลลัพธ์ได้ดีกับรูปภาพที่ราบเรียบเป็นช่วง (piecewise smooth image) \cite{ref:defination-of-variation-inpaint}  ซึ่งภาพวาดศิลปะไทยนั้น เป็นภาพซึ่งมีลักษณะราบเรียบเป็นช่วงจึงเหมาะสำหรับใช้แม่แบบวิธี ROF ซึ่งเป็นวิธีการเชิงแปรผัน สำหรับการต่อเติมภาพ เนื่องจากในปัจจุบันนี้ยังมีงานศิลปะไทยเก่าจำนวนมากที่ต้องการได้รับการซ่อมแซมจึงเป็นการช่วยให้จิตรกรผู้มีหน้าที่ซ่อมแซมภาพ สามารถเห็นตัวอย่างภาพที่ผ่านการต่อเติม เพื่อให้จิตรกรสามารถวางแผนในการซ่อมแซมได้ หรือการเข้าชมพิพิธภัณฑ์ ก็อาจพัฒนาเป็นแอปพลิเคชันสำหรับโทรศัพท์มือถือ ที่สามารถใช้ส่องภาพงานศิลปะและแสดงศิลปะที่สมบูรณ์ขึ้นมาทางหน้าจอได้
		 \begin{figure}[H]
		 	\centering
		 	\begin{subfigure}{0.4\linewidth}
		 		\centering
		 		\includegraphics[width=0.4\linewidth]{images/show_peicewise/thaiart_gray.png}
		 		\caption{ภาพศิลปะไทยซึ่งเป็นเฉดสีเทา}
		 	\end{subfigure}
		 	\begin{subfigure}{0.4\linewidth}
		 		\centering
		 		\includegraphics[width=0.4\linewidth]{images/show_peicewise/thaiart_is_piecewise.png}
		 		\caption{ความเข้มของสีบริเวณคอลัมม์ที่ 200}
		 	\end{subfigure}				
		 \end{figure}
		 โดยจากตัวอย่างจะเป็นภาพวาด จิตรกรรม อุโบสถวัดคงคาราม ขนาด 512x512 ซึ่งถูกปรับให้เป็นเฉดสีเทา เมื่อนำข้อมูลความเข้มของสีบริเวณคอลัมม์ที่ 200 มาเขียนกราฟจะเห็นว่าบริเวณจุดที่ 100 และจุดที่ 420 มีความแตกต่างของความเข็มอย่างชัดเจน โดยแบ่งเป็นช่วง (0,100), (100,420) และ (420,512) ซึ่งมีลักษณะราบเรียบเป็นช่วงนั่นเอง
		 

		 \subsection{การประยุกต์ใช้สำหรับการลบบทบรรยายบนอนิเมะ}
		 \hspace{1cm}อนิเมะคือวิดีโอภาพวาดการ์ตูนสไตล์ญี่ปุ่น ซึ่งภาพวาดสไตล์ของอนิเมะเองก็เป็นภาพแบบราบเรียบเป็นช่วงจึงเหมาะกับการใช้แม่แบบวิธี ROF สำหรับการต่อเติมภาพ	ซึ่งอนิเมะนั้นเนื่องจากผ่านการแปลภาษามาแล้ว ในบางครั้งจะมีคำบรรยายแบบแข็งติดมาด้วย ซึ่งบทบรรยายแบบแข็งนั้นไม่สามารถเปิดและปิดได้ เมื่อผ่านการให้เสียงใหม่แล้วก็จะมีบทบรรยายนั้นติดมาด้วย จึงเหมาะที่จะประยุกต์ใช้ในการลบทบรรยายออก
		 
		 สำหรับการต่อเติมภาพนั้น โดยปกติจะเป็นต้องหาโดเมนต่อเติมด้วยตัวเองซึ่งสามารถทำได้สำหรับภาพ  1 ภาพ แต่สำหรับไฟล์วิดีโออนิเมะนั้น ส่วนใหญ่จะมีภาพจำนวน 24 ภาพต่อวินาที ซึ่งนั่นหมายความว่าการจะหาโดเมนต่อเติมด้วยตัวเองสำหรับ 24 ภาพต่อวินาทีเป็นเรื่องยาก การที่จะหาโดเมนต่อเติมซึ่งเป็นในส่วนคำบรรยายนั้นจึงเป็นเรื่องยาก จึงเป็นอีกความท้าทายที่จะต้องหาโดเมนต่อเติมแบบอัตโนมัติให้ได้
		 
		 \begin{figure}[H]
		 	\centering
		 	\begin{subfigure}{0.4\linewidth}
		 		\centering
		 		\includegraphics[width=0.4\linewidth]{images/show_peicewise/anime_gray.png}
		 		\caption{ภาพอนิเมะซึ่งเป็นเฉดสีเทา}
		 	\end{subfigure}
		 	\begin{subfigure}{0.4\linewidth}
		 		\centering
		 		\includegraphics[width=0.4\linewidth]{images/show_peicewise/anime_is_piecewise.png}
		 		\caption{ความเข้มของสีบริเวณคอลัมม์ที่ 256}
		 	\end{subfigure}				
		 \end{figure}
		 โดยจากตัวอย่างจะเป็นภาพอนิเมะ ขนาด 512x512 ซึ่งถูกปรับให้เป็นเฉดสีเทา เมื่อนำข้อมูลความเข้มของสีบริเวณคอลัมม์ที่ 256 มาเขียนกราฟจะเห็นว่าบริเวณจุดที่ 100,250,350,400,420,480 มีความแตกต่างของความเข็มอย่างชัดเจน โดยแบ่งเป็นช่วง (0,100), (100,250), (250,350), (400,420)  และ (480,512) ซึ่งมีลักษณะราบเรียบเป็นช่วงนั่นเอง

		 \subsection{ความเร็วของขั้นตอนวิธีเชิงตัวเลข}
		 \hspace{1cm}ซึ่งวิธีการต่อเติมภาพด้วยวิธี Split Bergman ที่มีอยู่เดิมสำหรับภาพขาวดำขนาด 256x256 จะใช้เวลาประมวลผล 1.86 วินาที ดังที่จะเห็นได้จากรูปที่\ref{image:inpaint-grayscale} และรูปที่เป็นภาพสีขนาด 256x256 จะใช้เวลา 6.72 วินาที ดังที่จะเห็นได้จากรูปที่ \ref{image:inpaint-color} โดยสำหรับวิดีที่มีจำนวนเฟรม 24 เฟรมในหนึ่งวินาทีและมีขนาดเป็น 1920x1080 จะต้องใช้เวลาหลายวินาทีในการต่อเติมเฟรม 1 ภาพ จึงทำให้เวลาการต่อเติมวิดีโอ 1 วินาที จำเป็นต้องใช้เวลาหลายนาทีในการต่อเติม
		 
		 \hspace{1cm}เพราะฉะนั้นแล้วผู้วิจัยจึงสนใจที่จะพัฒนาขั้นตอนวิธีเชิงตัวเลขให้สามารถต่อเติมภาพได้รวดเร็วยิ่งขึ้น ซึ่งวิธีการเชิงตัวเลขที่พัฒนาขึ้นสามารถประยุกต์ใช้ได้กับการซ่อมแซมภาพวาดศิลปะไทย และการลบบทบรรยายบนอนิเมะ นอกจากนี้ยังสามารถนำไปประยุกต์ใช้กับงานต่อเติมภาพอื่นๆ ได้อีกในอนาคต
		 
\section{วรรณกรรมและทฤษฎีบทที่เกี่ยวข้อง}
 \subsection{การวัดคุณภาพของภาพหลังจากการต่อเติม}
	\hspace{1cm} หลังจากการต่อเติมภาพแล้วจำเป็นต้องพิจารณาว่าการวัดคุณภาพของภาพที่ผ่านการต่อเติมดีมากน้อยเพียงใด โดยในวิจัยนี้จะสนใจคุณภาพของค่าในแต่ละพิกเซลที่ใกล้เคียงกับภาพต้นฉบับ และโครงสร้างโดยรวมที่ใกล้เคียงกับภาพต้นฉบับ โดยการวัดค่าดังต่อไปนี้
	
	\hspace{1cm}  Peak signal-to-noise ratio (PSNR) \cite{ref:PSNR} ใช้สำหรับวัดคุณภาพของภาพโดยเปรียบเทียบจากพิกเซลแต่ละพิกเซล โดยภาพที่มีความคล้ายต้นฉบับจะมีค่า PSNR เข้าใกล้อนันต์ หรือก็คือยิ่งมีค่ามากยิ่งคุณภาพดี ซึ่งสามารถคำนวณได้โดย
	$$ PSNR = 10 \cdot log_{10} ( \frac{{peak}^2}{\sqrt{MSE}} )$$
	
	เมื่อ $MSE$ คือ mean square error และ $peak$ คือค่าสูงสุดโดยประเภทของภาพ ซึ่งสำหรับงานที่จะพูดถึงต่อไปนี้ จะพิจารณาภาพเป็นฟังก์ชันที่มีความเข้มของภาพอยู่ในช่วง $ [0,1] $ จึงได้ว่า $peak$ มีค่าเป็น $1$

	\hspace{1cm} Structural similarity (SSIM)  \cite{ref:SSIM} ใช้สำหรับว่าวัดคุณภาพของภาพจากโครงสร้างของภาพ โดยพิจารณาว่าภาพนั้นมีโครงสร้างแตกต่างหรือคล้ายคลึงกับภาพต้นฉบับมากน้อยเพียงใด โดยมีค่าอยู่ระหว่าง 0 ถึง 1 หากทั้งสองภาพมีความคล้ายคลึงกันมากค่า SSIM  จะเข้าใกล้กับค่า 1 ซึ่ง SSIM นั้นสามารถคำนวณได้โดย
		$$ SSIM(x,y) = \frac{(2\mu_x\mu_y + c_1)(2\sigma_{xy} + c_2)}{(\mu_x^2+\mu_y^2+c_1)(\sigma_x^2+\sigma_y^2+c_2)}$$
	เมื่อ $x,y$ คือภาพที่ต้องการเปรียบเทียบ $\mu$ คือค่าเฉลี่ยของภาพ $\sigma^2$ คือค่าความแปรปรวนของภาพ $\sigma_{xy}$ คือความแปรปรวนร่วม $c_1 =  (0.01L)^2, c_2 = (0.03L)^2$ และ $L$ คือค่าสูงสุดโดยประเภทของภาพ ซึ่งสำหรับงานที่จะพูดถึงต่อไปนี้ จะพิจารณาภาพเป็นฟังก์ชันที่มีความเข้มของภาพอยู่ในช่วง $ [0,1] $ จึงได้ว่า $L$ มีค่าเป็น 1


\subsection{วิธีการทางคณิตศาสตร์สำหรับการต่อเติมภาพด้วยการแปรผัน}
ในการต่อเติมภาพเฉดสีเทาด้วยวิธีการเชิงแปรผัน เราพิจารณาภาพ

$$ u : \Omega \subset \mathbb{R}^2 \rightarrow V \subset [0,\infty) $$

เป็นฟังก์ชันต่อเนื่อง โดยที่ $ \mathbf{x} = (x,y) \in \Omega $ แทนพิกัดทางกายภาพ (physical position) ของภาพ $ u(\mathbf{x}) \in V $ แทนระดับความเข้มของภาพ (image intensity) ที่ $ \mathbf{x} $ และ $ \Omega $ แทนโดเมนของภาพ ซึ่งในที่นี้เราสามารถสมมติได้โดย $ \Omega = [0,n]^2 $ และ $ V = [0,1] $ เมื่อ $n>0$ เป็นจำนวนเต็มบวก ทั้งนี้ เราจะเรียกภาพ $u$ ที่นิยามข้างต้นว่าภาพเฉดสีเทา (grayscale image)
\begin{figure}[H]
	\centering
	\includegraphics[width=0.4\linewidth]{images/sample-domain.png}
	\caption{ตัวอย่าง โดเมนต่อเติม}
\end{figure}
ซึ่งสำหรับปัญหาการต่อเติมภาพโทนสีเทานั้น จะเรียกพื้นที่ซึ่งต้องการต่อเติมว่า โดเมนต่อเติม (inpainting domain) โดย $D$ เป็นโดเมนซึ่ง $D \subset \Omega$ 

\hspace{1cm}การต่อเติมภาพเฉดเทานี้ จะใช้การแปรผันรวม (total variation) ซึ่งถูกคิดค้นโดย Chan และ Shen \cite{ref:rof-inpaint-chan-shen} ซึ่งประยุกต์มาจากการแปรผันรวมเพื่อกำจัดสัญญาณรบกวนบนตัวแบบ ROF \cite{ref:ROF-template} ได้ว่าจะสามารถหาภาพที่ถูกต่อเติมอย่างเหมาะสม $u$ จะสามารถหาได้จาก

$$\min_{u} \{ \mathcal{J}(u)= \lambda \mathcal{D}(u,f)+  \mathcal{R}(u) \}$$

ซึ่ง $ \mathcal{D} $ คือพจน์สำหรับวัดค่าเหมาะสม เพื่อไม่ให้ภาพก่อนต่อเติมและหลังจากต่อเติมมีความแตกต่างกันมากเกินไป $ \mathcal{R} $ คือพจน์สำหรับการต่อเติมภาพ และ $ \lambda $  คือเปนพารามิเตอรเร็กกิวลารไรซเซชัน (regularization parameter) สำหรับกำหนดปริมาณของสัญญาณรบกวนที่ตองการกำจัดออก 

\hspace{1cm} Chan และ Shen ได้ทำการเสนอปัญหาการแปรผัน (variational problem) สำหรับการต่อเติมภาพไว้ว่า

$$\min_{u} \{ \mathcal{J}(u) = \frac{\lambda}{2} \int_{\Omega \textbackslash D} (u-z)^2 d\Omega +  \int_{\Omega}  |\bigtriangledown u|  d\Omega \}$$

สำหรับ $x \in \Omega$ ได้ว่า $\lambda$ ถูกกำหนดโดย

% ควรไปอยู่ด้านบน
$$ \lambda(x) \left \{ \begin{array}{ll}  \lambda & x \in \Omega \textbackslash D \\ 0 & x \in D  \end{array} \right . $$

ซึ่งเมื่อทำให้ได้สมการออยเลอร์ลากรางซ์ สำหรับ fuctional $\mathcal{J}$ คือ 

$$ \left \{ \begin{array}{ll}  - \bigtriangledown \cdot  \Big( \frac{\bigtriangledown u}{|\bigtriangledown u|} \Big) + \lambda (u-z) = 0  & \hspace{1cm} x \in \Omega = (0,n)^2 \\ \frac{\partial u}{\partial n} = 0 & \hspace{1cm} x \in \partial \Omega \end{array} \right . $$

จะเห็นได้ว่าเป็นสมการออยเลอร์ ลากรางซ์ดังกล่าวเป็น สมการอนุพันธ์ย่อยไม่เป็นเชิงเส้น จึงสามารถใช้วิธีเชิงตัวเลขด้วยวิธีต่างๆ ได้ดังนี้

\hspace{1cm}วิธีการ Explicit Time Marching  \cite{ref:ExplicitTimeMarching}  โดยแนวคิดของวิธีการนี้คือการแนะนําตัวแปรเวลาสังเคราะห์ (time artificial variable) จากนั้นหาคําตอบแบบสภาวะคงตัว (steady-state solution) ของสมการเชิงอนุพันธ์ย่อยไม่เป็นเชิงเส้นที่ขึ้นอยู่กับเวลา และเพื่อจะแก้ความไม่เป็นเชิงเส้นของสมการเชิงอนุพันธ์ย่อย จะสามารถใช้รูปแบบที่ชัดแจ้งของออยเลอร์ (Euler's explicit scheme) ที่กำหนดโดย
$$
u(x,t_{k+1})=u(x,t_{k})+\tau\left(\bigtriangledown \cdot\left(\frac{\bigtriangledown u (x,t_k)}{| \bigtriangledown u (x,t_k) | }\right) + \lambda(x)(u (x,t_k)-z(x)) \right)
$$
แต่เนื่องจากเมื่อ $| \bigtriangledown u ( x,t_k) \rvert = 0 $ จะทำให้ไม่สามารถคำนวณได้ เนื่องจากหารด้วย 0 ไม่ได้ จึงกำหนดให้ $| \bigtriangledown u (x,t_k) | = \sqrt{u_x^2+u_y^2 + \beta}$ สำหรับ $\beta > 0$ และ $\beta$ มีค่าใกล้ 0 เพื่อหลีกเลี่ยงเหตุการณ์ดังกล่าว

\hspace{1cm} โดยเมื่อ $\tau>0$ แทนขั้นเวลา (time step) ที่ได้จากการดิสครีตไทซ์โดเมนเวลา $[0,\infty)$ ซึ่งเห็นได้ว่าวิธีการเชิงตัวเลขดังกล่าวข้างต้นนั้นง่ายในการคํานวณ แต่การลู่เข้าสู่คําตอบที่เหมาะสมของปัญหา เชิงแปรผันค่อนข้างช้ามากเนื่องจากต้องใช้ $\tau$ ที่มีขนาดเล็กในการทำให้ลำดับของคำตอบลู่เข้า 

\hspace{1cm}วิธีการทำซ้ำแบบ Fixed Point \cite{ref:FixpointSolver} จะทำการแบ่งการทำซ้ำออกเป็น  2 ชั้น เรียกชั้นในกับชั้นนอก โดยการทำซ้ำชั้นใน เป็นการทำซ้ำแบบ Gauss-seidel เพื่อให้หาค่า $u$ เป็นลำดับถัดไป จากนั้นชั้นนอกจะเป็นการทำซ้ำแบบตรึงจุด (fixed point) เพื่อให้ค่า $u$ ลู่เข้าสู่ค่าที่ต้องการ 

\hspace{1cm}ซึ่งการทำซ้ำชั้นในจะหาค่า $u$ จากการใช้ Gauss-Seidel กับสมการ
$$
- \bigtriangledown\cdot\left(\frac{\bigtriangledown u^{[v+1]}}{{| \bigtriangledown u |}^{[v]} }\right) + \lambda(u^{[v+1]}-z)  = 0
$$

โดยที่ $v = 0,1,2,... $ ซึ่งคือค่าของจำนวนครั้งในการทำชั้นนอกที่ถูกทำไป และเช่นเดียวกับวิธี Explicit Time Marching เนื่องจากวิธีนี้เมื่อ $| \bigtriangledown u |  = 0 $ จะทำให้ไม่สามารถคำนวณได้ เนื่องจากไม่สามารถหารด้วย 0 ได้ จึงกำหนดให้ $| \bigtriangledown u | = \sqrt{u_x^2+u_y^2 + \beta}$ สำหรับ $\beta > 0$ และ $\beta$ มีค่าใกล้ 0 เพื่อหลีกเลี่ยงเหตุการณ์ดังกล่าว

% บอกว่าอยากได้ beta น้อยๆ แต่มีปัญหาอย่างไร
\hspace{1cm} ซึ่งทั้งวิธี Explicit Time Marching และวิธี Fixed Point จำเป็นต้องเพิ่ม $\beta$ เพื่อหลีกเลี่ยงการหารด้วย 0 ซึ่งแม้จะเลือกค่า $\beta$  ดังกล่าวให้ใกล้ 0 มากๆ ก็ยังทำให้เกิดค่าคลาดเคลื่อน  จึงได้มีการสร้างวิธี Split Bergman สำหรับแก้ปัญหานี้

%บอกจากจุดเริ่มต้นของปัญหา ต้องการกำจัด beta ที่ทำให้คลาดเคลื่อน
\hspace{1cm} วิธี Split Bergman \cite{ref:splitbergman-inpaint} โดยวิธีการนี้คือการแยกส่วนการดำเนินการ (spliting) และใช้การทำซ้ำ Bergman (bergman iteration)  เพื่อที่จะกำจัดตัวแปร $\beta$ ที่ทำให้เกิดการหารศูนย์ขึ้น วิธีนี้จะเป็นการแก้ปัญหาการแปรผัน โดยการเพิ่มตัวแปร $w$ ในการแก้ปัญหา ซึ่งจะได้ปัญหาการแปรผันดังต่อไปนี้แทน

$$\min_{u} \{ \mathcal{J}(u) = \frac{\lambda}{2} \int_{\Omega \textbackslash D} (u-z)^2 d\Omega +  \int_{\Omega}  |\bigtriangledown w|  d\Omega + \frac{\theta}{2} \int_{\Omega} (w - \bigtriangledown u + b) d\Omega \}$$

โดยเมื่อเพิ่มตัวแปร $w$ สำหรับช่วยใจการคำนวณแล้ว จะมีพจน์ $ \int_{\Omega}  |\bigtriangledown w|  d\Omega$ เพื่อบีบบังคับให้ $w$ ไม่เปลี่ยนแปลงผลของคำตอบ, $\theta$ คือค่าบวกใดๆ ซึ่งเกี่ยวข้องกับความแรงของการต่อเติมซึ่งไม่ควรเล็กหรือใหญ่เกินไปเพื่อให้สามารถลู่เข้าได้ดี $b$ คือตัวแปรช่วยสำหรับการทำซ้ำ Bergman ซึ่งปัญหาดังกล่าวสามารถแบ่งปัญหาได้เป็น 2 ส่วนคือ

ปัญหาย่อย $w$ โดยการคงค่า $u$ ไว้จะได้ว่าปัญหาย่อยคือ
$$\min_{u} \{ \mathcal{J}(u) =  \int_{\Omega}  |\bigtriangledown w|  d\Omega + \frac{\theta}{2} \int_{\Omega} (w - \bigtriangledown u + b) d\Omega \}$$

ซึ่งเมื่อทำการแก้ปัญหาย่อยนี้แล้วจะได้ว่า

$$ w_{i,j} = \frac{\bigtriangledown u_{i,j}  + b_{i,j} }{ | \bigtriangledown u_{i,j}  + b_{i,j} | } max \{  | \bigtriangledown u_{i,j}  + b_{i,j} | - \frac{1}{\theta} , 0\} $$

ปัญหาย่อย $u$ โดยการคงค่า $w$ ไว้จะได้ว่าปัญหาย่อยคือ

$$\min_{u} \{ \mathcal{J}(u) = \frac{\lambda}{2} \int_{\Omega \textbackslash D} (u-z)^2 d\Omega + \frac{\theta}{2} \int_{\Omega} (w - \bigtriangledown u + b) d\Omega \}$$

ซึ่งเมื่อทำการแก้ปัญหาย่อยนี้แล้วจะได้ว่า
$$ \frac{1}{\theta}\lambda u - \bigtriangleup u = \frac{1}{\theta} \lambda z - \bigtriangledown \cdot (w-b) $$

ส่วนตัวแปรช่วย $b$ มีค่าเริ่มต้นเป็น 0 จากนั้นทำการปรับค่าโดย

$$ b^{k+1} = b^k  + \bigtriangledown u - w $$

จึงได้ว่าวิธีการ Split Bergman มีการทำงานในภาพรวมเป็นดังนี้



\begin{algorithm}[H]
	\begin{framed}
		initialization $u = 0, d = 0, b = 0$\\
		\While{ $|| u_{cur} - u_{prev} ||_2 > Tol$}{
			Solve the $w$ subproblem \\
			Solve the $u$ subproblem \\
			$ b = b + \bigtriangledown u - w$
		}
	\end{framed}
\end{algorithm}



การทำซ้ำนี้จะทำจนกระทั่ง นอร์ม L2 ระหว่างรอบปัจจุบันต่างกับรอบก่อนหน้าไม่เกินค่า Tol ที่กำหนดไว้หรือจำนวนรอบการทำซ้ำมากจนถึงจุดสิ้นสุดที่เพียงพอที่จะให้ลู่เข้าซึ่งไม่ควรใหญ่เกินไปเพื่อไม่ให้เสียเวลาประมวลผลจนนานเกินไป 

\hspace{1cm}ผลการต่อเติมภาพจากทั้ง 3 วิธีข้างต้น สำหรับการกำหนดรอบการทำซ้ำไม่เกิน 1000 รอบและค่านอร์ม L2 ภาพปัจจุบันและภาพก่อนหน้าต่างกันไม่เกิน 0.0001 ได้ผลลัพธ์ดังนี้

\begin{figure}[H]
	\centering
	\begin{subfigure}{0.3\linewidth}
		\centering
		\includegraphics[width=0.3\linewidth]{images/grayscale_inpaint/original.png}
		\caption{ภาพต้นฉบับ}
	\end{subfigure}
	\begin{subfigure}{0.3\linewidth}
		\centering
		\includegraphics[width=0.3\linewidth]{images/grayscale_inpaint/toinapint.png}
		\caption{ภาพที่ต้องการทำการต่อเติม}
	\end{subfigure}
	\begin{subfigure}{0.3\linewidth}
		\centering
		\includegraphics[width=0.3\linewidth]{images/grayscale_inpaint/inpaintdomain.png}
		\caption{โดเมนต่อเติม}
	\end{subfigure}
	\begin{subfigure}{0.3\linewidth}
		\centering
		\includegraphics[width=0.3\linewidth]{images/grayscale_inpaint/result_timemarch.png}
		\caption{วิธี Explicit Time Marching ใช้เวลา 3.10 วินาที PSNR 19.6733 SSIM 0.9463}
	\end{subfigure}
	\begin{subfigure}{0.3\linewidth}
		\centering
		\includegraphics[width=0.3\linewidth]{images/grayscale_inpaint/result_fixpoint.png}
		\caption{วิธี Fixed Point ใช้เวลา 6.93 วินาที PSNR 42.6631 SSIM 0.9869}
	\end{subfigure}
	\begin{subfigure}{0.3\linewidth}
		\centering
		\includegraphics[width=0.3\linewidth]{images/grayscale_inpaint/result_splitbergman.png}
		\caption{วิธี Split Bergman ใช้เวลา 1.86 วินาที PSNR 44.4275 SSIM 0.9965}
	\end{subfigure}
	\caption{การต่อเติมรูปภาพเฉดเทา}
	\label{image:inpaint-grayscale}
\end{figure}

	\hspace{1cm}จากการทดลองจะเห็นว่าวิธีการ Split Bergman ได้ผลลัพธ์ที่ดีที่สุด จึงได้มีความสนใจที่จะศึกษาวิธี Split Bergman เป็นลำดับถัดไป


% ต้องบอกว่า PSNR SSIM RMSE มาอย่างไร

\subsection{วิธีการทางคณิตศาสตร์สำหรับการต่อเติมภาพด้วยการแปรผันบนภาพสี} 

สำหรับการต่อเติมภาพเชิงแปรผันซึ่งเป็นภาพสีในระบบ RGB จะพิจารณา

$$ u = (u_1,u_2,u_3)^{\top} : \Omega  \rightarrow V^3 $$

\noindent เมื่อ $\u=(u_1,u_2,u_3)$ โดยที่ $u_1,u_2,u_3: \Omega  \rightarrow V$ แทนภาพเฉดสีแดง สีเขียว และสีน้ำเงินของ $u$ ตามลำดับ จะสามารถปรับปรุงตัวแบบ ROF ได้ดังนี้

$$\min_{u} \{ \mathcal{J}(u)= \lambda \mathcal{\bar{D}}(u,z)+  \mathcal{\bar{R}}(u) \}$$

โดยที่ $\mathcal{\bar{D}}$ เป็นพจน์วัดค่าเหมาะสมเพื่อไม่ให้ภาพก่อนการต่อและหลังต่อเติมต่างกันมาเกินไปซึ่ง
\begin{align*}
\mathcal{\bar{D}}(\boldsymbol{u},\boldsymbol{z})  &= \underset{l=1}{\overset{3}{\sum}}\mathcal{D}(u_l,z_l) \\
&= \frac{1}{2}\int_{\Omega}^{}(u_1 - z_1)^2 d\Omega + \frac{1}{2}\int_{\Omega}^{}(u_2 - z_2)^2 d\Omega + \frac{1}{2}\int_{\Omega}^{}(u_3 - z_3)^2 d\Omega
\end{align*}

และ $ \mathcal{\bar{R}} $ คือพจน์สำหรับการต่อเติมภาพ 

\begin{align*}
\mathcal{\bar{R}}(\boldsymbol{u}) &= \underset{l=1}{\overset{3}{\sum}}\mathcal{R}(u_l) = \int_{\Omega}^{}\lvert\nabla u_1 \rvert d\Omega + \int_{\Omega}^{}\lvert\nabla u_2 \rvert d\Omega + \int_{\Omega}^{}\lvert\nabla u_3 \rvert d\Omega
\end{align*}

ซึ่งงานนี้ได้สนใจที่จะนำวิธี Split Bergman มาประยุกต์ใช้กับปัญหาเชิงแปรผัน จึงแนะนำเวกเตอร์เสริม $w_1, w_2, w_3$ พร้อมทั้งตัวแปรสำหรับการทำซ้ำ Bregman $b_1, b_2, b_3$ และตัวแปรช่วย $\theta_1, \theta_2, \theta_3 > 0 $ เพื่อแก้ปัญหาเชิงแปรผัน ซึ่งจะได้ปัญหาเชิงแปรผันเป็น

$$ 
\min_{u} \{\bar{\mathcal{J}}= \lambda \mathcal{\bar{D}}(u,z) +  \underset{l=1}{\overset{3}{\sum}} \int_{\Omega}^{}|\boldsymbol{w_l}|d\Omega
+ \frac{\theta_l}{2} \underset{l=1}{\overset{3}{\sum}}\int_{\Omega}^{}(\boldsymbol{w_l} - \nabla u_l - \boldsymbol{b_l})^{2}d\Omega\}
$$

% เขียน functional เป็นเฉพาะ w เท่านั้น ไม่ต้องพูดอะไรมาก ยังไม่ต้องบง gauss-siedel เอา gauss-seidel ออกไปก่อน


ปัญหาย่อย $w$ โดยการคงค่า $u$ ไว้จะได้ว่าปัญหาย่อยคือ

$$ 
\min_{u} \{\bar{\mathcal{J}}= \underset{l=1}{\overset{3}{\sum}} \int_{\Omega}^{}|{w_l}|d\Omega
+ \frac{\theta_l}{2} \underset{l=1}{\overset{3}{\sum}}\int_{\Omega}^{}({w_l} - \nabla u_l - {b_l})^{2}d\Omega\}
$$

ซึ่งเมื่อทำการแก้ปัญหาย่อยนี้จะได้คำตอบที่แม่นตรงคือ

$$ w_{l_{i,j}} = \frac{\bigtriangledown u_{l_{i,j}}  + b_{l_{i,j}} }{ | \bigtriangledown u_{l_{i,j}}  + b_{l_{i,j}} | } max \{  | \bigtriangledown u_{l_{i,j}}  + b_{l_{i,j}} | - \frac{1}{\theta} , 0\} \hspace{1cm}  \hspace{0.1cm} l = 1,2,3$$ 


ปัญหาย่อย $u$ โดยการคงค่า $w$ ไว้จะได้ว่าปัญหาย่อยคือ

$$ 
\min_{u} \{\bar{\mathcal{J}}= \lambda \mathcal{\bar{D}}(u,z) + \frac{\theta_l}{2} \underset{l=1}{\overset{3}{\sum}}\int_{\Omega}^{}(\boldsymbol{w_l} - \nabla u_l - \boldsymbol{b_l})^{2}d\Omega\}
$$

ซึ่งเมื่อทำการแก้ปัญหาย่อยนี้แล้วจะได้ระบบสมการออยเลอร์ลากรางจ์คือ

$$ \frac{1}{\theta}\lambda u_1 - \bigtriangleup u_1 = \frac{1}{\theta} \lambda z_1 - \bigtriangledown \cdot (w_1-b_1) $$

$$ \frac{1}{\theta}\lambda u_2 - \bigtriangleup u_2 = \frac{1}{\theta} \lambda z_2 - \bigtriangledown \cdot (w_2-b_2) $$

$$ \frac{1}{\theta}\lambda u_3 - \bigtriangleup u_3 = \frac{1}{\theta} \lambda z_3 - \bigtriangledown \cdot (w_3-b_3) $$

ซึ่งเมื่อแก้ระบบสมการดังกล่าวแล้วจะนำค่าที่ได้มาเปลี่ยนค่าตัวแปร $b$ โดยที่

$$ b_{l}^{k+1} = b_{l}^k  + \bigtriangledown u_{l} - w_{l} \hspace{1cm}  \hspace{0.1cm} l = 1,2,3 $$

ซึ่งจากวิธีการ Split Bergman สำหรับภาพสี จะได้ภาพผลลัพธ์ดังนี้

\begin{figure}[H]
	\centering
	\begin{subfigure}{0.4\linewidth}
		\centering
		\includegraphics[width=0.4\linewidth]{images/color_inpaint/original.png}
		\caption{ภาพต้นฉบับ}
	\end{subfigure}
	\begin{subfigure}{0.4\linewidth}
		\centering
		\includegraphics[width=0.4\linewidth]{images/color_inpaint/toinpaint.png}
		\caption{ภาพที่ต้องการทำการต่อเติม}
	\end{subfigure}
	\begin{subfigure}{0.4\linewidth}
		\centering
		\includegraphics[width=0.4\linewidth]{images/color_inpaint/inpaintdomain.png}
		\caption{โดเมนต่อเติม}
	\end{subfigure}
	\begin{subfigure}{0.4\linewidth}
		\centering
		\includegraphics[width=0.4\linewidth]{images/color_inpaint/result_splitbergman.png}
		\caption{วิธี Split Bergman ใช้เวลา 6.72 วินาที PSNR 37.2374 SSIM 0.9976}
	\end{subfigure}
	\caption{การต่อเติมรูปภาพสี}
	\label{image:inpaint-color}
\end{figure}

\section{วัตถุประสงค์โครงการวิจัย}
วัตถุประสงค์ของโครงการวิจัยมีดังต่อไปนี้
\begin{description}
	\item[(1)]	 ศึกษาวิธีการแปรผันและวิธีการเชิงตัวเลขที่มีประสิทธิภาพเพื่อเติมข้อมูลที่ขาดหายในภาพหรือวิดีโอ
	\item[(2)] สร้างวิธีการเชิงตัวเลขใหม่สำหรับซ่อมแซมภาพศิลปะไทยและลบบทบรรยายออกจากอนิเมะ
	\item[(3)] นำวิธีการที่สร้างขึ้นเพื่อซ่อมแซมภาพไทย และลบบทบรรยายในอนิเมะ
\end{description}



\section{ขอบเขตการศึกษา}

% จะใช้ PSNR SSIM ของภาพ ของวิดีโออย่างไร
ขอบเขตของโครงงานมีดังต่อไปนี้
\begin{description}
	\item[(4.1)] ภาพศิลปะที่ใช้ศึกษา เป็นภาพจิตรกรรมฝาผนังไทย ที่อยู่ภายใต้เว็บ Wikipedia.org ซึ่งได้รับการอนุญาตให้ใช้งานแบบ Creative Commons หรือแบบ Public Domain
	\item[(4.2)] วิดีโอที่ใช้ศึกษาเป็นวิดีโอประเภทอนิเมะ โดยศึกษากับไฟล์อนิเมะที่ใช้ Color space แบบ RGB เท่านั้น
	\item[(4.3)] บทบรรยายที่ใช้ทดสอบ จะถูกล้อมรอบไว้ด้วยสีดำ ขนาดความหนาขนาดไม่น้อยกว่า 5 พิกเซล
	\item[(4.4)] วิดีโอที่ใช้ศึกษาขนาดไม่เกิน 1920x1080
	\item[(4.5)] คอมพิวเตอร์ที่ใช้ทดลองใช้หน่วยประมวลผล I7-6700HQ ใช้การ์ดจอ Nvidia GTX 960M แรม 16GB ฮาร์ดดิกส์แบบ SSD
	\item[(4.6)] จะศึกษาคุณภาพของภาพหลังจากผ่านการต่อเติมโดยใช้ค่า PSNR และ SSIM โดยสำหรับไฟล์วิดีโอนั้นจะใช้การเฉลี่ยจากจำนวนเฟรมทั้งหมดในวิดี หากเฟรมใดได้ค่า PSNR เป็นอนันต์ จะทำการข้ามเฟรมนั้นไป
\end{description}

\section{ระเบียบวิธีวิจัย}
วิธีการมีดังต่อไปนี้
\begin{description}
	\item[(5.1)] ศึกษาการคณิตศาสตร์ต่อเติมข้อมูลที่ขาดหายบนรูปภาพ
	\item[(5.2)] พัฒนาวิธีการเชิงตัวเลขสำหรับการซ่อมแซมรูปภาพ
	\item[(5.3)] ทดสอบวิธีการเชิงตัวเลขที่พัฒนาขึ้นโดยโปรแกรมคอมพิวเตอร์บนภาพสังเคราะห์
	\item[(5.4)] อภิปรายผลที่ได้จากการทดลองเชิงตัวเลข
	\item[(5.5)] สรุปผลการดำเนินงานวิจัยและจัดทำรูปเล่มฉบับสมบูรณ์
\end{description}

\section{แผนการดำเนินงานวิจัย}
แผนการดำเนินงานตลอดทั้งโครงการสามารถสรุปได้โดยย่อจากตารางต่อไปนี้
\begin{center}
	\begin{tabular}[ht]{|l|c|c|c|c|c|c|c|c|c|c|c|c|}
		\hline
		&\multicolumn{12}{c|}{เดือนที่}\\
		\cline{2-13}
		แผนการดำเนินงาน&1&2&3&4&5&6&7&8&9&10&11&12\\
		\hline
		ศึกษาการคณิตศาสตร์ต่อเติมข้อมูลที่ขาดหายบนรูปภาพ&x&x& & & & & & & & & &\\
		พัฒนาวิธีการเชิงตัวเลขสำหรับการซ่อมแซมรูปภาพ& & &x&x& & & & & & & &\\
		ทดสอบวิธีการเชิงตัวเลขที่พัฒนาขึ้นโดยโปรแกรม- & & & & &x&x& & & & & &\\
		คอมพิวเตอร์บนภาพสังเคราะห์ & & & & & & & & & & & &\\
		อภิปรายผลที่ได้จากการทดลองเชิงตัวเลข & & & & & & &x&x& & & &\\
		สรุปผลการดำเนินงานวิจัยและจัดทำรูปเล่มฉบับสมบูรณ์& & & & & & & & &x&x&x&x\\
		\hline
	\end{tabular}
\end{center}



\section{บรรณานุกรม}

\renewcommand{\section}[2]{} % addition mod: remove reference text
\begin{thebibliography}{}
	\bibitem{ref:defination-of-inpaint} 
	Borko Furht., Encyclopedia of Multimedia. Springer. pp. 315-316, 2006. %\url{https://doi.org/10.1007/0-387-30038-4_98}
	
	\bibitem{ref:defination-of-variation-inpaint}
	Işık Barış Fidaner. “A Survey on Variational Image Inpainting , Texture Synthesis and Image Completion.” 2007. 
	\url{https://www.semanticscholar.org/paper/_/36f4d32ce45f72091510ab4d4d1cc3bf81ffe879}
	
	\bibitem{ref:PSNR}
	David Salomon. Data Compression: The Complete Reference (4 ed.). Springer. pp. 281. 2007.
	%ISBN 978-1846286025.  \url{https://books.google.co.th/books?id=ujnQogzx_2EC&lpg=PA281&ots=FolwqB8qsN&dq=PSNR+infinite&pg=PA281&redir_esc=y#v=onepage&q=PSNR%20infinite&f=false}
	
	\bibitem{ref:SSIM}
	Zhou Wang, Alan Conrad Bovik, Hamid Rahim Sheikh and Eero P. Simoncelli, "Image quality assessment: from error visibility to structural similarity," in IEEE Transactions on Image Processing, vol. 13, no. 4, pp. 600-612, 2004.
	%\url{https://doi.org/10.1109/TIP.2003.819861}
	
	\bibitem{ref:rof-inpaint-chan-shen} 
 	Tony F. Chan and Jianhong Shen , “Mathematical models of local non-texture inpaintings”, SIAM Journal on Applied Mathematics, vol. 62, no. 3, pp. 1019–1043, 2001. 
	%\url{http://www.jstor.org/stable/3061798}
	
	\bibitem{ref:ROF-template} 
	Leonid I. Rudin, Stanley Osher, Emad Fatemi, "Nonlinear total variation based noise removal algorithms", Physica D: Nonlinear Phenomena, Volume 60, Issues 1–4, pp. 259-268, 1992. 
	%\url{https://doi.org/10.1016/0167-2789(92)90242-F}
	
	\bibitem{ref:ExplicitTimeMarching} 
	Antonio Marquina and Stanley Osher, "Explicit algorithms for a new time dependent model based on level set motion for nonlinear deblurring and noise removal",  
	SIAM Journal on Scientific Computing, vol. 22 issue 2, 387–405. 2006. 
	% \url{https://doi.org/10.1137/S1064827599351751}
	
	\bibitem{ref:FixpointSolver}
	Curtis R. Vogel and M.E. Oman,"Iterative methods for total variation denoising", SIAM Journal on Scientific Computing. vol. 17, pp. 227-238, 1996.
	
	\bibitem{ref:splitbergman-inpaint}
	Pascal Getreuer, Total Variation Inpainting using Split Bregman, Image Processing On Line, 2 (2012), pp. 147–157. 
	%\url{https://doi.org/10.5201/ipol.2012.g-tvi}

	
\end{thebibliography}

\end{document}
	
	
	
	