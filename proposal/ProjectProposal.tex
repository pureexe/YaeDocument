\documentclass[hidelinks,a4paper,14pt]{article}
\usepackage{ amsmath, amssymb}
\usepackage{graphicx}
\usepackage{color,amsmath,graphics,graphicx}
\usepackage{amsfonts}
\usepackage{mathrsfs,hyperref}
\usepackage{latexsym,amsmath,enumerate,amsbsy,amsthm}
\textwidth = 415pt

%==============================================
\usepackage{fontspec}
\usepackage{xunicode}
\usepackage{xltxtra}
\defaultfontfeatures{Scale=1.23}
\XeTeXlinebreaklocale “th_TH” % สำหรับตัดคำ
\setmainfont[Scale=1.23]{TH SarabunPSK}
%==============================================
%%%%%%%%%%%%%%% THEOREM Environments %%%%%%%%%% 
\newtheorem{theorem}{ทฤษฎีบท}[section]						%
\newtheorem{lemma}[theorem]{บทตั้ง}						%
\newtheorem{conjecture}[theorem]{บทคาดการณ์}				%
\newtheorem{definition}[theorem]{บทนิยาม}					%
\newtheorem{remark}[theorem]{หมายเหตุ}						%
\newtheorem{proposition}[theorem]{ประพจน์}					%
\newtheorem{corollary}[theorem]{บทแทรก}					%
\numberwithin{equation}{section}							%
\newtheorem{example}[theorem]{ตัวอย่าง}						%
%\newtheorem{exercise}{แบบฝึกหัด}[chapter]	
%\renewcommand{\chaptername}{บทที่}
\renewcommand\tablename{ตารางที่}
\renewcommand\figurename{รูปที่}
\renewcommand{\contentsname}{สารบัญ}						%
%\renewcommand{\bibname}{บรรณานุกรม}						% 
\renewcommand{\indexname}{ดรรชนี}					%
%%%%%%%%%%%%%%%%%%%%%%%%%%%%%%%%%%%%%%%%%%%%%%%

\begin{document}
	{\begin{center}
			\textbf{Project Proposal}\\
			\vspace{0.5cm}
			\textbf{Division of Applied Mathematics, Department of Mathematics}\\
			\vspace{0.5cm}
			\textbf{Faculty of Science, Silpakorn University}
		\end{center}}
		{
			\vspace{0.5cm}
			\flushleft{Date : { 27 กันยายน 2561 } \hfill{ }}
			\flushleft{Advisor : {  ผู้ช่วยศาสตราจารย์ ดร. นพดล  ชุมชอบ}}\\
			\flushleft{Student : { นายภัคพล       พงษ์ทวี     รหัส  07580028}\\
			\vspace{1cm}
		}
		
		
		% Here the project title
		{\textbf{\begin{flushleft}Project Title : ลบบทบรรยายแบบแข็งบนวิดีโอแบบอนิเมะ \\
				(Hard-coded subtitle remover for anime)
			\end{flushleft}
		}}
		\thispagestyle{empty}
	\section{Introduction}
	\setlength\parindent{24pt}\hspace{\parindent} %MOD: Tab
	 บทบรรยายใต้ภาพ (subtitle) คือข้อความที่ได้จากการถอดเสียงจาก ภาพยนตร์ รายการทีวี หรือสื่อต่างๆ ที่มักแสดงอยู่ที่ด้านล่างของหน้าจอ ซึ่งใช้เพื่อแสดงคำแปลในภาษาอื่น หรือเพื่อช่วยเหลือผู้ประสบปัญหาทางการได้ยิน โดยบทบรรยายใต้ภาพ แบ่งตามการสร้างได้เป็น 2 ประเภทได้แก่ แบบ Soft และแบบ Hard โดยแบบ Soft จะมีลักษณะเป็นข้อความที่ถูกระบุตำแหน่งเวลาไว้ สามารถเปิดและปิดได้ แต่ก็มีปัญหาว่าเครื่องเล่นหรือซอร์ฟแวร์ที่ใช้จำเป็นต้องรองรับรูปแบบคำบรรยายนั้นจึงจะแสดงขึ้นมาได้ และในบางครั้งก็มีปัญหาด้านการเข้ารหัส ทำให้ไม่สามารถแสดงข้อความได้อย่าวถูกต้องและแบบ Hard คือ ข้อความในบทบรรยายจะถูกฝังรวมเป็นเนื้อเดียวกับวิดีโอ ซึึงวิธีนี้จะทำให้ไม่จำเป็นต้องใช้เครื่องเล่นหรือซอร์ฟแวร์พิเศษเพื่อรองรับ ทำให้ไม่มีปัญหาในการแสดงผล แต่ว่าบทบรรยายแบบนี้จะไม่สามารถทำการเปิดและปิดได้\newline 
	
	จากการที่บทบรรยายใต้ภาพแบบ Hard ไม่สามารถเปิดและปิดได้ทำให้บทบรรยายดังกล่าวก่อกวนผู้รับชม ตัวอย่างเช่น ภาพยนตร์จีนกำลังภายในที่ฉายทางโทรทัศน์ไทยมักจะมีบทบรรยายภาษาจีนติดมาด้วย เนื่องจากประเทศจีนมีการใช้ภาษาจีนในสำเนียงที่หลากหลาย จึงมีบทบรรยายให้สามารถรับชมได้โดยเข้าใจความหมายที่เหมือนกัน โดยคำบทบรรยายที่ใช้นั้น เป็นบทบรรยายแบบ Hard ซึ่งเมื่อภาพยนตร์ดังกล่าวได้รับการให้เสียงไทยแล้ว ภาพยนตร์จึงมีคำบรรยายภาษาจีนติดมาด้วย \newline
	
	ด้วยปัญหาที่กล่าวมาข้างต้นโดยผู้ศึกษาได้มีความสนใจที่จะลบคำบรรยายแบบ Hard ออกจากวิดีโอ โดยเฉพาะวิดิโอแบบอนิเมะ เนื่องจากวิดีโอประเภทนี้มักมีบทบรรยายติดมาด้วย และบางครั้ง มีบทบรรยายเป็นแบบ Hard ทำให้ไม่สามารถเปิดและปิดได้ \newline
	
	วิดีโอนั้นประกอบด้วยภาพจำนวนหลายภาพต่อหนึ่งหน่วยเวลา เราจะเรียกภาพหนึ่งภาพในวิดีโอว่า เฟรม (frame) ซึ่งภาพในแต่ละเฟรมเรียกว่าภาพดิจิตัล (Digital Image) โดยภาพดิจิตัลนั้นจะสามารถนิยามได้เป็นฟังก์ชัน $f(x,y)$ โดยที่ $x$ และ $y$ เป็นพิกัดของภาพ และแอมพิจูดของ $f$ ที่พิกัด $(x,y)$ ใดๆ ในภาพคือค่าความเข้มแสง (intensity)  ซึ่งแอมพิจูดนี้มีค่าจำกัด\newline
	
	การซ่อมแซมภาพ คือกระบวนการที่จะเติมเต็มข้อมูลที่หายไปในพื้นที่ภาพที่กำหนด โดยมีจุดประสงค์เพื่อซ่อมแซมภาพที่เสียหาย โดยพื้นที่ภาพส่วนนั้นไม่สามารถพบได้จากการสังเกต โดยการกู้คืน สี, โครงสร้าง และพื้นผิว ที่เกิดการเสียหายเป็นวงกว้าง พิกเซลที่จะนำมาใช้ซ่อมแซมจะถูกคำนวณขึ้นมาใหม่จากข้อมูลที่พิกเซลที่อยู่โดยรอบที่ยังไม่เสียหาย ซึ่งการจะนำบทบรรยายออกจากเฟรมวิดีโอนั้น จะพิจารณาว่าบทบรรยายนั้นเป็นส่วนที่เสียหาย แล้วจากนั้นจึงใช้การซ่อมแซมภาพเพื่อนำบทบรรยายนั้นออก\newline
	
	การซ่อมแซมภาพมีวิธีการทางคณิตศาสตร์ที่ใช้แตกต่างกันไปจำนวนมาก แต่เนื่องจากภาพอนิเมะที่เราต้องการซ่อมแซมนั้นเป็นรูปภาพที่ราบเรียบเป็นช่วง (piecewise smooth image) จึงเหมาะที่จะใช้วิธีการซ่อมภาพด้วยการแปรผัน (Variation) 
	
	โดยโมเดลทางคณิตศาสตร์ที่จะใช้งานในการซ่อมแซมรูปภาพนี้คือ การแปรผันรวม (Total Variation)  ซึ่งม่ีที่มาจากปัญหา Rudin-Osher-Fatemi (ROF) โดยเป็นการแก้ปัญหาการแปรผันมีขอบเขต (bounded variation หรือ BV) ทั้งหมดโดยที่ภาพ $u$ อยู่ใน $BV(\Omega)$ เมื่อสามารถหาปริพันธ์ได้และจะมี Radon measure $Du$ ซึ่ง 
	
	$$\int_{\Omega}u(x) div \vec{g}(x) dx = \int_{\Omega} \left\langle\vec{g},Du(x) \right\rangle\hspace{1cm}\forall\vec{g} \in C_c^1(\Omega,\mathbb{R}^2)^2$$
	
	และจาก $Du$ เป็น distributional gradient ของ $u$ เมื่อ $u$ ราบเรียบแล้ว  $Du(x)= \bigtriangledown u(x)dx $
	โดย total variation seminorm ของ $u$ คือ 
	
	$$ ||u||_{TV(\Omega)} := \int_{\Omega} | Du | := sup{ \left \{ \int_{\Omega}  u \ div  \ \vec{g} \ dx \  : \vec{g} \  \in C_c^1(\Omega,\mathbb{R}^2)^2 \ , \ \sqrt{g_1^2+g_2^2} \leq 1 \right \} }  $$
	
	จาก $u$ ราบเรียบแล้ว การแปรผันรวมสมมูลกับอินทิกรัลของขนาดเกรเดียนท์ 
	
	$$ ||u||_{TV(\Omega)} = \int_{\Omega} | \bigtriangledown u | dx$$
	
	จึงได้ว่าจะหาฟังก์ชันแปรผันมีขอบเขต $u$ หาได้จาก minimization problem
	
	$$ \underset{u \in BV(\Omega)}{arg \ min} ||u||_{TV(\Omega)} + \frac{\lambda}{2} \int_{\Omega \textbackslash D} (f(x) - u(x))^2 dx$$
	
	เมื่อ $\lambda$ มีค่าบวก ปัญหา minimalization นี้จะเหมือนกับปัญหาการลบสิ่งรบกวนของ Rudin, Osher และ Fatemi เพียงแต่ปริพันธ์ลำดับอยู่บน $\Omega-D$  แทนที่จะเป็น $\Omega$ ถ้าผลลัพธ์ที่แม่นตรงอยู่ใน $BV$ และมีค่าอยู่ในช่วง $[0,1]$ แล้วจะมี minimizer u แต่มักจะไม่มีเพียงหนึ่งเดียว
	
	การซ่อมแซมรูปภาพอาจมองเป็นลักษณะการลบสิ่งรบกวนที่มี spatially-varying regularization strength เป็น $\lambda(x)$ ทำให้ได้ว่า
	
		$$\underset{u}{{arg \ min}} ||u||_{TV(\Omega)} + \frac{1}{2} \int_{\Omega} \lambda(x)(f(x) - u(x))^2 dx$$
		
	โดยที่ $\lambda(x)$ จะมีค่าเป็น $0$ เมื่ออยู่ใน $D$ และ $\lambda(x)>0$ เมื่ออยู่นอก $D$  ทำให้เมื่อ $x \in D$ ที่ $\lambda(x)=0$ ค่า $f(x)$ จะไม่ถูกใช้ ทำให้ $u(x)$ ได้รับผลจาก $||u||_{tv}$ เท่านั้น ส่วนที่ด้านนอก $D$ จะเป็น TV-regularize denoising พฤติกรรมลดสิ่งลบกวนนี้อาจเป็นที่น่าพอใจเมื่อยากที่จะระบุโดเมนที่ต้องซ่อมแซมได้อย่างถูกต้อง และเมื่อใช้ $λ$ ขนาดใหญ่จะทำให้การลดสิ่งรบกวนมีผลน้อยมากจนทำให้พื้นที่นอก $D$  แทบไม่เปลี่ยนแปลง
	
	จากโมเดลทางคณิตศาสตร์ที่ได้กล่าวมาข้างต้น จะสามารถใช้วิธีการทางเชิงตัวเลขสำหรับการซ่อมแซมรูปภาพโดยใช้ความแปรปรวนทั้งหมด
	
	จากความแปรปรวนทั้งหมดสามารถประมาณได้โดย $ |\bigtriangledown u_{i,j} | $ บนทุกพิกเซลนั่นคือ
	
	$$ ||u||_{TV(\Omega)} near by $$
	
	เมื่อ $\bigtriangledown u_{i,j}$  คือ discrete gradient วิธี split bergman คือการแยกส่วนการดำเนินการ (splitting) และการทำซ้ำ bergman (bergman iteration) ซึ่งวิธี split bergman จะนำมาใช้เพื่อแก้ minimization problem

$$ arg min / d = \bigtriangledown u $$

	โดยตัวแปรช่วย $d$  คือเวคเตอร์ที่บีบบังคับ $ \bigtriangledown u$ และใช้วิธีการทำซ้ำ bergman เพื่อแก้ปัญหาค่าเหมาะสมแบบมีข้อจำกัด ซึ่งในแต่ละการทำซ้ำ bergman จะเป็นการแก้
	
	$$ arg min sum $$
	
	เมื่อ $b$  เป็นตัวแปรของวิธีการทำซ้ำ bergman และ $\gamma$ เป็นค่าคงที่บวกใดๆ โดยการ minimization  บน $d$ และ $u$  จะแก้โดย alternative direction method โดยแต่ละขั้นของการหาค่าต่ำสุด ตัวแปร $d$ และ $u$ จะให้ตัวแปรอื่นคงค่าไว้
	
	d subproblem เมื่อเราคงค่า u ไว้ จะได้ว่า d subproblem คือ
	
	$$ arg min sum $$
	
	โดยปัญหานี้เมื่อทำการแก้แล้วจะได้ว่า 
	
	$$ d_{i,j} = $$
	
	u subproblem เมื่อเราคงค่า $d$ ไว้ จะได้ว่า u subproblem คือ
	
	$$ arg min $$
	
	เมื่อแก้แล้วจะได้ว่า
	
	$$ 1 / gamma$$
	
	โดยที่ $div$ คือ discrete divergence และ $\bigtriangledown u$ คือ discrete lapacian เราจะประมาณคำตอบนี้โดยการใช้ หนึ่งรอบ Gauss-seidel ต่อหนึ่งรอบการทำซ้ำของ Bergman ซึ่ง subproblem จะถูกแก้หนึ่งครั้ง ต่อหนึ่งรอบ bergman iteration แต่ทั้งนี้การทำซ้ำ Gauss-seidel หลายครั้ง จะทำให้การแก้ subproblem มีความแม่นยำขึ้น
	ส่วนตัวแปรช่วย $b$ มีค่าเริ่มต้นเป็น 0 จากนั้นทำการปรับค่าโดย
	
	$$ b^{k+1} = b^k  + \bigtriangledown u - d $$
	
	โดยที่ความเกี่ยวข้องกันของแต่ละพื้นที่จะแรงขึ้นเมื่อ γ ใหญ่ขึ้น ดังนั้น γ ไม่ควรเล็กหรือใหญ่จนเกินไป จะทำให้ทั้งสอง subproblem ลู่เข้าได้ดี
	จึงได้ว่าวิธีการในภาพรวมเป็นดังนี้
	
	% algorithm insertion
	
	โดยการทำซ้ำนี้จะกระทำจนกระทั่ง นอร์ม L2 ระหว่างรอบปัจจุบันต่างกับรอบก่อนหน้าไม่เกินค่า Tol ที่กำหนดไว้หรือจำนวนรอบการทำซ้ำมากจนถึงจุดสิ้นสุดที่เพียงพอที่จะให้ลู่เข้าซึ่งไม่ควรใหญ่เกินไปเพื่อไม่ให้เสียเวลาประมวลผลจนนานเกินไป
	
	

\section{Objective}
วัตถุประสงค์ของโครงการวิจัยมีดังต่อไปนี้
\begin{description}
	\item[(1)] เพื่อลบคำบรรยายแบบ Hard ออกจากวิดีโออนิเมะ
\end{description}

\section{Scope of Study}
ขอบเขตของโครงงานมีดังต่อไปนี้
\begin{description}
\item[(3.1)] วิดีโอที่ใช้ศึกษาเป็นวิดีโอประเภทอนิเมะเท่านั้น
\item[(3.2)] บทบรรยายที่ใช้ทดสอบ จะถูกล้อมรอบไว้ด้วยสีดำ ขนาดความหนาขนาดไม่น้อยกว่า 5 พิกเซล
\item[(3.3)] จะทำการทดสอบการลบคำบรรยายบน 4 ภาษาได้แก้ ไทย จีน อังกฤษและญี่ปุ่น
\end{description}

\section{Methodology}
วิธีการมีดังต่อไปนี้
\begin{description}
	\item[(4.1)] ????
	\item[(4.9)] เขียนรายงานโครงการวิจัย
\end{description}
\section{Time Periods}
แผนการดำเนินงานตลอดทั้งโครงการสามารถสรุปได้โดยย่อจากตารางต่อไปนี้
\begin{center}
	\begin{tabular}[ht]{|l|c|c|c|c|c|c|c|c|c|c|c|c|}
		\hline
		&\multicolumn{12}{c|}{เดือนที่}\\
		\cline{2-13}
		แผนการดำเนินงาน&1&2&3&4&5&6&7&8&9&10&11&12\\
		\hline
		ศึกษาเนื้อหาหัวข้อเรขาคณิต&x&x& & & & & & & & & &\\
		เขียนรายงานโครงการวิจัย& & & & & & & & & & & &x\\
		\hline
	\end{tabular}
\end{center}




\section{References}
พิมพ์เอกสารอ้าอิงในหัวข้อนี้
\end{document}
	
	
	
	