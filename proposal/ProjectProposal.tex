\documentclass[hidelinks,a4paper,14pt]{article}
\usepackage{ amsmath, amssymb}
\usepackage{graphicx}
\usepackage{color,amsmath,graphics,graphicx}
\usepackage{amsfonts}
\usepackage{mathrsfs,hyperref}
\usepackage{latexsym,amsmath,enumerate,amsbsy,amsthm}
\textwidth = 415pt

%==============================================
\usepackage{fontspec}
\usepackage{xunicode}
\usepackage{xltxtra}
\defaultfontfeatures{Scale=1.23}
\XeTeXlinebreaklocale “th_TH” % สำหรับตัดคำ
\setmainfont[Scale=1.23]{TH SarabunPSK}
%==============================================
%%%%%%%%%%%%%%% THEOREM Environments %%%%%%%%%% 
\newtheorem{theorem}{ทฤษฎีบท}[section]						%
\newtheorem{lemma}[theorem]{บทตั้ง}						%
\newtheorem{conjecture}[theorem]{บทคาดการณ์}				%
\newtheorem{definition}[theorem]{บทนิยาม}					%
\newtheorem{remark}[theorem]{หมายเหตุ}						%
\newtheorem{proposition}[theorem]{ประพจน์}					%
\newtheorem{corollary}[theorem]{บทแทรก}					%
\numberwithin{equation}{section}							%
\newtheorem{example}[theorem]{ตัวอย่าง}						%
%\newtheorem{exercise}{แบบฝึกหัด}[chapter]	
%\renewcommand{\chaptername}{บทที่}
\renewcommand\tablename{ตารางที่}
\renewcommand\figurename{รูปที่}
\renewcommand{\contentsname}{สารบัญ}						%
%\renewcommand{\bibname}{บรรณานุกรม}						% 
\renewcommand{\indexname}{ดรรชนี}					%
%%%%%%%%%%%%%%%%%%%%%%%%%%%%%%%%%%%%%%%%%%%%%%%

\begin{document}
	{\begin{center}
			\textbf{Project Proposal}\\
			\vspace{0.5cm}
			\textbf{Division of Applied Mathematics, Department of Mathematics}\\
			\vspace{0.5cm}
			\textbf{Faculty of Science, Silpakorn University}
		\end{center}}
		{
			\vspace{0.5cm}
			\flushleft{Date : { 14 lb'sk8, 2561 } \hfill{ }}
			\flushleft{Advisor : {  ผู้ช่วยศาสตราจารย์ ดร. พรทรัพย์  พรสวัสดิ์}}\\
			\flushleft{Student(s) : { 1. นางสาวชญานี       นนทศักดิ์     รหัส  07550629}\\
				\hspace{0.7in}{ 2. นางสาวนารีรัตน์   ประสมสาสตร์   รหัส  07550635}}
			\vspace{1cm}
		}
		
		
		% Here the project title
		{\textbf{\begin{flushleft}Project Title : ผลการใช้บทเรียนโอริงามิต่อนักเรียนชั้นมัธยมศึกษาปีที่ 2 \\
				(The effect of using origami lessons on the eight – grade students.)
			\end{flushleft}
		}}
		\thispagestyle{empty}
	\section{Introduction}
	ระบบการศึกษาทั่วทุกมุมโลกแปรผันตรงต่อการขับเคลื่อนของวิวัฒนาการโลก ซึ่งมีการ
เปลี่ยนแปลงอย่างรวดเร็วตามยุคสมัย สถานการณ์โลกในศตวรรษที่21 แตกต่างจากศตวรรษที่
20 และ19 เป็นอย่างมาก มีการพัฒนาและเปลี่ยนแปลงในทุกๆ ด้านไม่ว่าจะเป็นด้านเศรษฐกิจ
สังคม วิทยาศาสตร์ และเทคโนโลยี รวมถึงสิ่งแวดล้อม ก่อให้เกิดผลกระทบโดยตรงต่อการจัด
ระบบการศึกษาในทุกๆ ระดับของประเทศ และยังส่งผลให้ทั่วโลกมุ่งพัฒนาบุคลากรให้เป็นผู้ที่
มีความพร้อมทางด้านความรู้ และความสามารถทันต่อการเปลี่ยนแปลงของโลกในยุคปัจจุบัน
ในศตวรรษที่ 21 เป็นยุคแห่งความเจริญก้าวหน้าของเทคโนโลยีสารสนเทศ \\

ความรู้นอกห้องเรียนทุกวันนี้มีการเพิ่มขึ้นอย่างรวดเร็วผ่านสื่อเทคโนโลยีต่างๆ ซึ่งสังคมปัจจุบันไม่
ได้ต้องการผู้เรียนที่รู้เยอะ ไม่ได้ต้องการผู้เรียนที่ท่องจำเก่ง เรียนเก่งเพียงอย่างเดียว แต่จะ
ต้องเป็นผู้เรียนที่ใฝ่รู้ อยากที่จะเรียนรู้สิ่งใหม่ๆ ตลอดเวลา ควบคู่ไปกับการรู้วิธีการเรียนรู้
สิ่งเหล่านั้นด้วยความเข้าใจ นั่นคือทักษะกระบวนการเรียนรู้ที่จำเป็น จุดสำคัญในศตวรรษนี้
คือต้องเปลี่ยนวิธีการของการศึกษา เปลี่ยนเป็าหมายจาก “ความรู้” ไปสู่ “ทักษะ” เปลี่ยน
จาก “ครู” เป็นหลัก เน้นบทบาท “ผู้เรียน” เป็นหลัก คือเรียนรู้โดยการลงมือปฏิบัติจริง สนใจ...{\textbf{ให้นักศึกษาปรับเปลี่ยนบทนำในส่วนนี้}}\\

\section{Objective}
วัตถุประสงค์ของโครงการวิจัยมีดังต่อไปนี้

\begin{description}
	\item[(1)] เพื่อศึกษาการพัฒนาทักษะทางด้านการมองมิติสัมพันธ์โดยใช้สื่อโอริงามิ
	\item[(2)] เพื่อออกแบบสื่อหรือคู่มือการสอนสำหรับการพัฒนาทักษะการมองมิติสัมพันธ์
	\item[(3)]  เพื่อทดสอบประสิทธิภาพของผู้ใช้สื่อหรือคู่มือการสอน
\end{description}

\section{Scope of Study}
ขอบเขตของโครงงานมีดังต่อไปนี้
\begin{description}
\item[(3.1)] ขอบเขตเนื้อหา
เนื้อหาหัวข้อเรขาคณิตในหนังสือเรียนรายวิชาคณิตศาสตร์ชั้นมัธยมศึกษาปีที่2 ตามหลักสูตรแกนกลางการศึกษาขั้นพื้นฐาน พุทธศักราช 2551 ของกระทรวงศึกษาธิการ
\item[(3.2)]  ประชากรและตัวอย่าง\\
ประชากร คือ นักเรียนชั้นมัธยมตอนต้น ปีการศึกษา 2558\\
ตัวอย่างที่ใช้ในโครงงานวิจัยคือ นักเรียนชั้นมัธยมศึกษาปีที่ 2 ปีีการศึกษา 2558 ในโรงเรียนที่สนใจเข้าร่วมทำกิจกรรม ตามคู่มือปฏิบัติการที่จัดทำขึ้น ซึ่งคู่มือปฏิบัติการจะสอดคล้องกับเนื้อหาหัวข้อเรขาคณิต ในหนังสือรายวิชาคณิตศาสตร์ ชั้นมัธยมศึกษาตอนต้น ตามหลักสูตรแกนกลางการศึกษาขั้นพื้นฐาน พุทธศักราช 2551 ของกระทรวงศึกษาธิการ
\item[(3.3)]  ตัวแปรที่จะศึกษา\\
ตัวแปรอิสระคือ คู่มือปฏิบัติการโดยใช้สื่อโอริงามิ\\
ตัวแปรตามคือ พัฒนาการทางด้านการมองมิติสัมพันธ์
\item[(3.4)]  สมมติฐานการวิจัย
ทักษะทางด้านการมองมิติสัมพันธ์ของนักเรียนชั้นมัธยมศึกษาปี‚ที่ 2 หลังการใช้สื่อหรือคู่มือการสอน มีการพัฒนาขึ้น
\item[(3.5)]  เครื่องมือที่ใช้ในการวิเคราะห์\\
เครื่องมือที่ใช้ในการวิเคราะห์ประกอบด้วย
\begin{itemize}
	\item สื่อหรือคู่มือการสอนโอริงามิสำหรับนักเรียนชั้นมัธยมศึกษาปีที่ 2
	\item แบบทดสอบทักษะทางด้านการมองมิติสัมพันธ์ (ใช้วิธีการเก็บข้อมูลแบบ One-group
	Pretest-Posttest Design)
\end{itemize}
สถิติที่ใช้ในการวิเคราะห์ข้อมูล ได้แก่ ประสิทธิภาพของสื่อโอริงามิ ค่าเฉลี่ย และส่วนเบี่ยงเบนมาตรฐาน
\end{description}

\section{Methodology}
วิธีการมีดังต่อไปนี้
\begin{description}
	\item[(4.1)] ศึกษาเนื้อหาหัวข้อเรขาคณิตในหนังสือเรียนรายวิชาคณิตศาสตร์ ชั้นมัธยมศึกษาปีที่ 2
ตามหลักสูตรแกนกลางการศึกษาขั้นพื้นฐาน พุทธศักราช 2551 ของกระทรวงศึกษาธิการ
\item[(4.2)] ศึกษาและวิเคราะห์การพับกระดาษแบบโอรงามิจากหนังสือโอริงามิพับกระดาษ ฝึกสมอง
และโอริงามิกระดาษพับ อัศจรรย์เล่ม 2 ของ ดร.บัญชา ธนบุญสมบัติ
\item[(4.3)] ออกแบบคู่มือการปฏิบัติการ 2 ชุดโดยแบ่งเป็นคู่มือกิจกรรม และเอกสารกิจกรรม ให้
สอดคล้องกับ หัวข้อเรขาคณิต ชั้นมัธยมศึกษาปีที่2 ตามหลักสูตรแกนกลางการศึกษาขั้น
พื้นฐาน พุทธศักราช 2551
\item[(4.4)] ออกแบบทดสอบก่อนการใช้คู่มือปฏิบัติการและหลังใช้คู่มือปฏิบัติการ
\item[(4.5)] ให้ผู้เชี่ยวชาญตรวจสอบ จำนวน 5 ท่าน
\item[(4.6)] ติดต่อประสานงานกับทางโรงเรียนเพื่อมีการจัดกิจกรรมค่าย โดยผู้เข้าร่วมกิจกรรมเป็น
นักเรียนชั้นมัธยมศึกษาปีที่ 2
\item[(4.7)] เก็บข้อมูลการวิจัยโดยการทำแบบทดสอบ
\item[(4.8)] วิเคราะห์ข้อมูลจากการทำแบบทดสอบ
\item[(4.9)] เขียนรายงานโครงการวิจัย
\end{description}
\section{Time Periods}
แผนการดำเนินงานตลอดทั้งโครงการสามารถสรุปได้โดยย่อจากตารางต่อไปนี้
\begin{center}
	\begin{tabular}[ht]{|l|c|c|c|c|c|c|c|c|c|c|c|c|}
		\hline
		&\multicolumn{12}{c|}{เดือนที่}\\
		\cline{2-13}
		แผนการดำเนินงาน&1&2&3&4&5&6&7&8&9&10&11&12\\
		\hline
		ศึกษาเนื้อหาหัวข้อเรขาคณิต&x&x& & & & & & & & & &\\
		ศึกษาและวิเคราะห์การพับกระดาษแบบโอรงามิ&x&x& & & & & & & & & &\\
		ออกแบบคู่มือการปฏิบัติการ 2 ชุด& &x&x& & & & & & & & &\\
		ออกแบบทดสอบก่อน-หลังการใช้คู่มือปฏิบัติการ& & & &x&x& & & & & & &\\
		ให้ผู้เชี่ยวชาญตรวจสอบ& & & & & &x&x& & & & &\\
		ติดต่อประสานงานกับทางโรงเรียน& & & & & &x&x& & & & &\\
		เก็บข้อมูลการวิจัย& & & & & & & &x&x& & &\\
		วิเคราะห์ข้อมูลจากการทำแบบทดสอบ& & & & & & & & & &x&x&\\
		เขียนรายงานโครงการวิจัย& & & & & & & & & & &x&x\\
		รายงานวิจัยฉบับสมบูรณ์& & & & & & & & & & & &x\\
		\hline
	\end{tabular}
\end{center}




\section{References}
พิมพ์เอกสารอ้าอิงในหัวข้อนี้
\end{document}
	
	
	
	