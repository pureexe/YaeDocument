% !TEX program = xelatex
\documentclass[hidelinks,a4paper,14pt]{article}
\usepackage{ amsmath, amssymb}
\usepackage{graphicx}
\usepackage{color,amsmath,graphics,graphicx}
\usepackage{amsfonts}
\usepackage{mathrsfs,hyperref}
\usepackage{latexsym,amsmath,enumerate,amsbsy,amsthm}
\textwidth = 415pt

%==============================================
\usepackage{fontspec}
\usepackage{xunicode}
\usepackage{xltxtra}
\defaultfontfeatures{Scale=1.23}
\XeTeXlinebreaklocale “th_TH” % สำหรับตัดคำ
\setmainfont[Scale=1.23]{TH SarabunPSK}
%==============================================
\usepackage{arp}
\begin{document}
	\arptitle{9 เมษายน 2562} % Input the date here.
	{4} % The number of your record
	{ผศ.ดร. นพดล ชุมชอบ}% Input the name of your advisor here.
	{นาย ภัคพล พงษ์ทวี} %  Add your name here.
	 
	\ptitle{ขั้นตอนวิธีเชิงตัวเลขชนิดใหม่สำหรับการต่อเติมภาพที่ใช้การแปรผันรวมกับการประยุกต์สำหรับซ่อมแซมภาพจิตรกรรมไทยโบราณและการลบบทบรรยายจากอนิเมะ \\ (A new numerical algorithm for TV-based image inpainting with its applications for restoring ancient Thai painting images and removing subtitles from animes)}
	\pprogress{\quad}{9 เมษายน 2562}
	\pwork{เสร็จสิ้นการอภิปรายผลที่ได้จากการทดลองเชิงตัวเลขและการสรุปผลการดำเนินการวิจัย พร้อมทั้งเริ่มจัดทำรายงานวิจัยฉบับสมบูรณ์}
	
	\acomment{\quad}
	\vfill
	\bottom
\end{document}
