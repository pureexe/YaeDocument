\documentclass[a4paper,12pt,oneside]{book}
\usepackage{appreport}
\usepackage{amsmath,amsthm,amssymb}
\usepackage{graphicx}
\linespread{1.5}
\usepackage{makeidx} 
\makeindex
%==============================================
\usepackage{fontspec}
\usepackage{xunicode}
\usepackage{xltxtra}
\defaultfontfeatures{Scale=1.23}
\XeTeXlinebreaklocale “th_TH” % สำหรับตัดคำ
\setmainfont[Scale=1.23]{THSarabunPSK}
%==============================================
\renewcommand{\chaptername}{บทที่}
\makeatother
\newtheorem{Theorem}{\bf ทฤษฎีบท}[chapter]
\newtheorem{Definition}{\bf  บทนิยาม}
\newtheorem{Lemma}{ \bf ทฤษฎีบทประกอบ}[section]   
\newtheorem{Proposition}{\bf ประพจน์}[section]
\newtheorem{Remark}{\bf หมายเหตุ}[section]
\newtheorem{Corollary}{\bf บทแทรก}[section]
\newtheorem{Example}{\bf ตัวอย่าง}[section]
\newtheorem{Exercise}{\bf แบบฝึกหัด}[chapter]
\newtheorem{Observation}{\bf ข้อสังเกต}[section]
\renewcommand{\contentsname}{สารบัญ}
\renewcommand{\bibname}{บรรณานุกรม}
\renewcommand{\indexname}{ดัชนี}
\renewcommand\listfigurename{สารบัญรูป}
\renewcommand\listtablename{สารบัญตาราง}
\renewcommand{\appendixname}{ภาคผนวก}
\begin{document}
\Rtitle{การจัดตารางสอนและตารางสอบของภาควิชาคณิตศาสตร์ มหาวิทยาลัยศิลปากรด้วยขั้นตอนเชิงพันธุกรรม}{Creating Class Schedule and Exam Schedule for Department of Mathematics,	Silpakorn University, by using Genetic Algorithm}%
{ภัทรพร  พูลสวัสดิ์ รหัส 07560024\\ ฮัซลามีย์  เบ็ญจวงค์   รหัส 07560643}{อาจารย์ ดร. นัยน์รัตน์ กันยะมี}

\thispagestyle{empty}

\clearpage
\backcover{สาขาวิชาคณิตศาสตร์ประยุกต์ ภาควิชาคณิตศาสตร์ คณะวิทยาศาสตร์ มหาวิทยาลัยศิลปากร มีความเห็นชอบให้โครงงานวิจันเรื่องการจัดตารางสอนและตารางสอบของภาควิชาคณิตศาสตร์ มหาวิทยาลัยศิลปากรด้วยขั้นตอนเชิงพันธุกรรม (Creating Class Schedule and Exam Schedule for Department of Mathematics,	Silpakorn University, by using Genetic Algorithm) 
ซึ่งเสนอโดย นางสาวภัทรพร  พูลสวัสดิ์ รหัส 07560024 และนางสาว ฮัซลามีย์  เบ็ญจวงค์   รหัส 07560643 เป็นส่วนหนึ่งของการศึกษาตามหลักสูตรวิทยาศาสตรบัณฑิต สาขาวิชาคณิตศาสตร์ประยุกต์ ประจำปีการศึกษา 2559}{อาจารย์ ดร. นัยน์รัตน์ กันยะมี}{ผู้ช่วยศาสตราจารย์ ดร. พรทรัพย์  พรสวัสดิ์}{อาจารย์ ดร. ภาสวรรณ  นพแก้ว}

\newpage
\pagenumbering{roman}
\newpage
\thispagestyle{empty}
\vspace{2 cm}
{\huge \bf บทคัดย่อ}\addcontentsline{toc}{chapter}{บทคัดย่อ}

\vspace{2 cm}
ขั้นตอนวิธีเชิงพันธุกรรม (Genetic algorithm)  เป็นเทคนิคทางปัญญาประดิษฐ์ที่ลอกเลียนแบบมา
จากวิวัฒนาการตามธรรมชาติเพื่อใช้ในการค้นหาคำตอบที่ดีที่สุด ย้อนกลับไปเมื่อปี 1859 ชาลส์ดาร์วิน
(Charles Darwin) ได้เสนอทฤษฎีวิวัฒนาการตามธรรมชาติ ซึ่งกล่าวถึงความหลากหลายและความแตกต่าง
ระหว่างสิ่งมีชีวิตที่อยู่รวมกันในธรรมชาติ เนื่องจากสภาพแวดล้อมที่เปลี่ยนไปส่งผลต่อการดำรงชีวิตของสิ่ง
มีชีวิต  สิ่งมีชีวิตที่สามารถปรับตัวเข้ากับสภาพแวดล้อมได้ดีกว่าจะสามารถอยู่รอดในธรรมชาติ และสามารถถ่ายทอดลักษณะทางพันธุกรรมไปสู่ลูกหลานในรุ่นต่อๆ ไป ในขณะเดียวกันสิ่งมีชีวิตที่ไม่สามารถปรับตัวให้เข้า
กับสภาพแวดล้อมจะค่อยๆ ลดจำนวนลงและสูญพันธ์ลงไปในที่สุด ต่อมา เกรเกอร์เมนเดล (Gregor Mendel)
ได้ทำการทดลองปลูกพืชตระกูลถั่วเพื่อศึกษาลักษณะเด่นและลักษณะด้อยในรุ่นพ่อแม่ที่ถูกถ่ายทอดไปยังรุ่น
ลูก จึงเป็นที่มาของทฤษฎีที่ว่าด้วยการถ่ายทอดลักษณะทางพันธุกรรม กระบวนการทางธรรมชาติ

\vspace{1 cm}
{\bf{คำสำคัญ:}} พันธุกรรม ปัญญาประดิษฐ์\\

\newpage
\thispagestyle{empty}
\vspace{2 cm}
{\huge \bf Abstract}\addcontentsline{toc}{chapter}{บทคัดย่อ}

\vspace{2 cm}
Genetic algorithm

\vspace{1 cm}
{\bf{Keywords:}} Genetic \\
\include{ack}
\clearpage
\thispagestyle{empty}
\tableofcontents
\listoffigures\addcontentsline{toc}{chapter}{สารบัญรูป}
 % to produce list of figures 
\listoftables \addcontentsline{toc}{chapter}{สารบัญตาราง}
% to produce list of tables

\clearpage
\pagenumbering{arabic}

\chapter{บทนำ}
ขั้นตอนวิธีเชิงพันธุกรรม (Genetic algorithm) \cite{nontiya} เป็นเทคนิคทางปัญญาประดิษฐ์ที่ลอกเลียนแบบมา
จากวิวัฒนาการตามธรรมชาติเพื่อใช้ในการค้นหาคำตอบที่ดีที่สุด ย้อนกลับไปเมื่อปีค.ศ.1859 ชาลส์ดาร์วิน
(Charles Darwin) ได้เสนอทฤษฎีวิวัฒนาการตามธรรมชาติ ซึ่งกล่าวถึงความหลากหลายและความแตกต่าง
ระหว่างสิ่งมีชีวิตที่อยู่รวมกันในธรรมชาติ เนื่องจากสภาพแวดล้อมที่เปลี่ยนไปส่งผลต่อการดำรงชีวิตของสิ่ง
มีชีวิต สิ่งมีชีวิตที่สามารถปรับตัวเข้ากับสภาพแวดล้อมได้ดีกว่าจะสามารถอยู่รอดในธรรมชาติ และสามารถ
ถ่ายทอดลักษณะทางพันธุกรรมไปสู่ลูกหลานในรุ่นต่อๆ ไป ในขณะเดียวกันสิ่งมีชีวิตที่ไม่สามารถปรับตัวให้เข้า
กับสภาพแวดล้อมจะค่อยๆ ลดจำนวนลงและสูญพันธ์ลงไปในที่สุด ต่อมา เกรเกอร์เมนเดล (Gregor Mendel)
ได้ทำการทดลองปลูกพืชตระกูลถั่วเพื่อศึกษาลักษณะเด่นและลักษณะด้อยในรุ่นพ่อแม่ที่ถูกถ่ายทอดไปยังรุ่น
ลูก จึงเป็นที่มาของทฤษฎีที่ว่าด้วยการถ่ายทอดลักษณะทางพันธุกรรม กระบวนการทางธรรมชาติ

\chapter{ขั้นตอนวิธีเชิงพันธุกรรม}
ขั้นตอนวิธีเชิงพันธุกรรมมีองค์ประกอบดังนี้
\begin{description}
	\item[(1)] การสร้างประชากรเริ่มต้น (Initial Population)
	\item[(2)] การประเมินค่าความเหมาะสม (Evaluation)
	\item[(3)] การคัดเลือก (Selection)
	\item[(4)] การสลับสายพันธุ์(Crossover)
	\item[(5)] การกลายพันธุ์(Mutation) 
\end{description}
ในกระบวนการวิธีเชิงพันธุกรรม เราจะทำการกำหนดประชากรเริ่มต้น เพื่อเป็นการง่ายเราจะอธิบาย
แต่ละขั้นตอนด้วยการยกตัวอย่างโดยสมมติโครโมโซมให้มียีนส์เป็นระบบเลขฐานสอง 
\section{การสร้างประชากรเริ่มต้น (Initial Population)}
ในขั้นตอนนี้จะกำหนดจำนวนประชากรเริ่มต้นหรือจำนวนโครโมโซมเริ่มต้น ซึ่งในที่นี้ถ้ากำหนดให้ประชากร
เริ่มต้นมีจำนวนโครโมโซม 6 โครโมโซม แต่ละโครโมโซมมี8 ยีนส์ระบบจะทำการสุ่มเลือกโครโมโซมมา 6 โครโมโซม
ตัวอย่างเช่น

\section{การประเมินค่าความเหมาะสม (Evaluation)}
เป็นขั้นตอนในการประเมินค่าของแต่ละโครโมโซมที่ได้จากการคำนวณผ่านฟังก์ชันความเหมาะสมที่เรา
สร้างขึ้นสำหรับปัญหาที่สนใจ เนื่องจากกำหนดให้ในที่นี้มีโครโมโซมเริ่มต้นจำนวน 6 โครโมโซม แต่ละโครโมโซม
มี 8 ยีนส์ซึ่งแต่ละยีนส์บรรจุเลขฐาน 2 คือ 0 และ 1 เราจะทำการประเมินค่าของฟังก์ชันโดยในที่นี่เรากำหนด
ให้เลข 1 ในยีนส์แทนค่าเท่ากับ 1 และเลข 0 ในยีนส์แทนค่าเท่ากับ 0

\chapter{การแก้ปัญหา}
\section{รายวิชาในหลักสูตรคณิตศาสตร์และคณิตศาสตร์ประยุกต์}
จากการสำรวจรายวิชาที่เรียนในหลักสูตรคณิตศาสตร์และคณิตศาสตร์ประยุกต์พบว่ามีรายวิชาทั้งหมด
67 รายวิชา สำหรับรายวิชาที่เปิดสอนโดยภาคคณิตศาสตร์ภาคการศึกษาที่1 ปีการศึกษา 2559 ในระดับปริญญา
ตรีมีทั้งหมด 29 รายวิชา ดังนี้
\section{การกำหนดโครโมโซม}
เมื่อเราได้จำนวนรายวิชาทั้งหมดแล้ว ต่อมาจะทำการสร้างสายโครโมโซมซึ่งประกอบด้วยจำนวนยีนส์ที่แทน
จำนวนรายวิชาทั้งหมดที่เรามีในแต่ละเทอม ซึ่งเราจะนำจำนวนรายวิชาที่นักศึกษาแต่ละปีต้องเรียนมาเรียงต่อกัน
ตามโครงสร้างที่เราได้กำหนดไว้โดยโครโมโซมที่เราจะพิจารณาในแต่ละครั้งจะแบ่งเป็น 6 ประเภท คือ
\section{การกำหนดค่าแอลิล (Allele)}
สำหรับการกำหนดค่าแอลิลหรือการกำหนดเลขยีนส์ในโครโมโซม \cite{wutt}  เราจะแบ่งพิจารณาเป็น 3 ประเภท ดังนี้
\subsection{ค่าแอลิลของเวลาเรียน}
เราทำการกำหนดค่าในแต่ละยีนส์ซึ่งแทนเวลาเรียนทั้งหมด 70 เลข ดังนี้โดยเลข 1 ถึง 30 แทนเวลา
เรียนที่เรียนต่อกัน 2 คาบ และเลข 31 ถึง 70 แทนเวลาเรียน 1 คาบ ดังนี้
\subsection{ค่าแอลิลของเวลาสอบกลางภาค}
เวลาของตารางสอบกลางภาคมีทั้งหมด 18 เลข ดังนี้
\chapter{ผลการวิจัย}
ยีนส์ในตำแหน่งที่5 คือรายวิชา 514101 General Physics I ชนกับ ยีนส์ตำแหน่งที่33 คือรายวิชา 519482
Selected Topics in Applied Analysis จึงทำให้มีคะแนน 200,000 คะแนน เนื่องจากเงื่อนไขข้อ (7) ที่ว่า
ตัวเลขบนยีนส์ในตำแหน่งของวิชา 514101 General Physics I ตรงกับวิชาในภาคคณิตศาสตร์
- ยีนส์ในตำแหน่งที่29 คือรายวิชา 511441 Abstract Algebra II ชนกับยีนส์ตำแหน่งที่34 คือรายวิชา 519481
Mathematical Models in the Biological Sciences ทำให้มีคะแนน 100 คะแนน เนื่องจากเกณฑ์การให้
คะแนนระดับ 3 ข้อ (1) ที่ว่า วิชาในกลุ่ม 1.1 ชนกับวิชาในกลุ่ม 3.1 หรือกลุ่ม 4.1 ส่วนยีนส์ในตำแหน่งอื่นๆ ที่
มีการซ้ำของเลขในตำแหน่งยีนส์เราไม่ได้นำมาคิดคะแนน เพราะไม่ได้กำหนดเงื่อนไขของยีนส์ตำแหน่งนั้น
\chapter{สรุป}
\addcontentsline{toc}{chapter}{บรรณานุกรม}

\bibliographystyle{plain}
\bibliography{myref}
\clearpage

\appendix
\chapter{ภาคผนวก}
\addcontentsline{toc}{chapter}{ภาคผนวก}

	แอแอแอืหก่าด่กด
	ฃืำไยฟเำ
	\newpage
	dfdfdf
\chapter{พันธุกรรม}
\addcontentsline{toc}{chapter}{ภาคผนวก A}
\chapter{ตารางสอน}
\addcontentsline{toc}{chapter}{ภาคผนวก B}

\end{document}


