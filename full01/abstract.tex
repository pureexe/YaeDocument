\pagenumbering{roman}
\newpage
\thispagestyle{empty}
\vspace{2 cm}
{\huge \bf บทคัดย่อ}\addcontentsline{toc}{chapter}{บทคัดย่อ}

\vspace{2 cm}
ขั้นตอนวิธีเชิงพันธุกรรม (Genetic algorithm)  เป็นเทคนิคทางปัญญาประดิษฐ์ที่ลอกเลียนแบบมา
จากวิวัฒนาการตามธรรมชาติเพื่อใช้ในการค้นหาคำตอบที่ดีที่สุด ย้อนกลับไปเมื่อปี 1859 ชาลส์ดาร์วิน
(Charles Darwin) ได้เสนอทฤษฎีวิวัฒนาการตามธรรมชาติ ซึ่งกล่าวถึงความหลากหลายและความแตกต่าง
ระหว่างสิ่งมีชีวิตที่อยู่รวมกันในธรรมชาติ เนื่องจากสภาพแวดล้อมที่เปลี่ยนไปส่งผลต่อการดำรงชีวิตของสิ่ง
มีชีวิต  สิ่งมีชีวิตที่สามารถปรับตัวเข้ากับสภาพแวดล้อมได้ดีกว่าจะสามารถอยู่รอดในธรรมชาติ และสามารถถ่ายทอดลักษณะทางพันธุกรรมไปสู่ลูกหลานในรุ่นต่อๆ ไป ในขณะเดียวกันสิ่งมีชีวิตที่ไม่สามารถปรับตัวให้เข้า
กับสภาพแวดล้อมจะค่อยๆ ลดจำนวนลงและสูญพันธ์ลงไปในที่สุด ต่อมา เกรเกอร์เมนเดล (Gregor Mendel)
ได้ทำการทดลองปลูกพืชตระกูลถั่วเพื่อศึกษาลักษณะเด่นและลักษณะด้อยในรุ่นพ่อแม่ที่ถูกถ่ายทอดไปยังรุ่น
ลูก จึงเป็นที่มาของทฤษฎีที่ว่าด้วยการถ่ายทอดลักษณะทางพันธุกรรม กระบวนการทางธรรมชาติ

\vspace{1 cm}
{\bf{คำสำคัญ:}} พันธุกรรม ปัญญาประดิษฐ์\\

\newpage
\thispagestyle{empty}
\vspace{2 cm}
{\huge \bf Abstract}\addcontentsline{toc}{chapter}{บทคัดย่อ}

\vspace{2 cm}
Genetic algorithm

\vspace{1 cm}
{\bf{Keywords:}} Genetic \\