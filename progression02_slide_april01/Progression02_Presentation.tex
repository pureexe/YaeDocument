% !TEX program = xelatex

\documentclass[xcolor=dvipsnames, xetex,serif]{beamer}
%\documentclass[handout,xetex,serif]{beamer} %ใช้บรรทัดนี้สำหรับปริ้นเอกสาร
\usepackage{color,amsmath,graphics,graphicx}
\usepackage{epsfig,amsfonts,graphics}
\usepackage{mathrsfs,hyperref}
\usepackage{subcaption,float,framed,algorithm2e,hyperref}
%===============================================
\usepackage{fontspec,xltxtra,xunicode}
\defaultfontfeatures{Scale=1.23}
\XeTeXlinebreaklocale “th_TH” % สำหรับตัดคำ
\setmainfont[Scale=1.23]{THSarabunNew}
% 1.23 เท่าคือจาก 12 pt บน LaTeX ให้เท่ากับ 16pt บน Word
%=====================================================
%\usepackage{pgfpages} %ใช้บรรทัดนี้สำหรับปริ้นเอกสาร
%\pgfpagesuselayout{4 on 1}[a4paper,border shrink=5mm,landscape]
%\pgfpagesuselayout{2 on 1}[a4paper,border shrink=5mm]
 %ใช้บรรทัดนี้สำหรับปริ้นเอกสาร

%%%%%%%%%%%%%%% THEOREM Environments %%%%%%%%%% 					
\newtheorem{conjecture}[theorem]{บทคาดการณ์}								
\newtheorem{remark}[theorem]{หมายเหตุ}										
\numberwithin{equation}{section}							
\renewcommand\tablename{ตารางที่}
\renewcommand\figurename{รูปที่}						
\renewcommand{\bibname}{บรรณานุกรม}						
\renewcommand{\indexname}{ดรรชนี}
\setbeamertemplate{caption}[numbered]	
\setbeamertemplate{theorems}[numbered]				
%%%%%%%%%%%%%%%%%%%%%%%%%%%%%%%%%%%%%%%%%%%%%%%

\mode<presentation>{
	\usetheme{Madrid}
	\usecolortheme[named=PineGreen]{structure}}
 \title[วิธีเชิงตัวเลขสำหรับต่อเติมภาพ]{\normalsize{ขั้นตอนวิธีเชิงตัวเลขชนิดใหม่สำหรับการต่อเติมภาพที่ใช้การแปรผันรวมกับการประยุกต์สำหรับซ่อมแซมภาพจิตรกรรมไทยโบราณและการลบบทบรรยายจากอนิเมะ\\A new numerical algorithm for TV-based image inpainting with its applications for restoring ancient Thai painting images and removing subtitles from animes}}
 \author[ภัคพล]{ภัคพล พงษ์ทวี}
 \institute[SU]{
 	ภาควิชาคณิตศาสตร์\\
 	มหาวิทยาลัยศิลปากร \\}
 \date[Project Progression]{การนำเสนอความก้าวหน้าโครงงานวิจัย ครั้งที่ 2\\
 	9 เมษายน 2562}
 
 \AtBeginSubsection[]{
 	\begin{frame}<beamer>
 		\frametitle{Outlines}
 		\tableofcontents [currentsection,currentsubsection]
	 \end{frame}
}
\begin{document}
	\begin{frame}
 		\titlepage 
	\end{frame}	 
	\begin{frame}
		\frametitle{ตัวแบบการต่อเติมภาพที่ใช้การแปรผันรวม}
		\begin{align*}
		\min_{u} \{ \mathcal{J}(u) = \frac{1}{2} \int_{\Omega}\lambda (u-z)^2 d\Omega +  \int_{\Omega}  |\nabla u|  d\Omega \}
		\end{align*}
		 \vspace{1cm}
		\begin{align*}
		\lambda=\lambda(\mathbf{x}) = \left \{ \begin{array}{ll}  \lambda_0, & x \in \Omega \textbackslash D \\ 0, & x \in D  \end{array} \right . 
		\end{align*}
		\let\thefootnote\relax\footnotetext{\tiny{T.F. Chan and J. Shen , “Mathematical models of local non-texture inpaintings”, SIAM Journal on Applied Mathematics, vol. 62, no. 3, pp. 1019–1043, 2001.}}	
	\end{frame} 
	\begin{frame}
		\frametitle{ขั้นตอนวิธีเชิงตัวเลข}
		\begin{align*}
			u(\mathbf{x},t_{k+1})=u(\mathbf{x},t_{k})+\tau\left(\nabla \cdot\left(\dfrac{\nabla u (\mathbf{x},t_k)}{| \nabla u (\mathbf{x},t_k) | }\right) + \lambda(\mathbf{x})(u (\mathbf{x},t_k)-z(\mathbf{x})) \right)
		\end{align*}
		\begin{center}
		\textcolor{PineGreen}{วิธีที่ 1:} การเดินเวลา (Explicit time marching)
		\end{center}
		\begin{align*}
			- \nabla\cdot\left(\dfrac{\nabla u^{[\nu+1]}}{{| \nabla u |}^{[v]} }\right) + \lambda(u^{[\nu+1]}-z)  = 0,\ u^{[0]}=z
		\end{align*}
		\begin{center}
			\textcolor{PineGreen}{วิธีที่ 2:} ทำซ้ำจุดตรึง (Fixed point iteration)
		\end{center}
		\begin{align*}
			\min_{u,\boldsymbol{w}} \{ \mathcal{J}(u,\boldsymbol{w}) = \dfrac{1}{2} \int_{\Omega} \lambda(u-z)^2 d\Omega +  \int_{\Omega}  |\boldsymbol{w}|  d\Omega + \frac{\theta}{2} \int_{\Omega} (\boldsymbol{w} - \nabla u + \boldsymbol{b}) d\Omega \}
		\end{align*}
		\begin{center}
			\textcolor{PineGreen}{วิธีที่ 3:} สปริทเบรกแมน (Split Bregman)
		\end{center}
	\end{frame} 
	\begin{frame}
		\frametitle{ปัญหาเชิงตัวเลข}
			\begin{figure}[H]
			\centering
			\includegraphics[width=0.2\linewidth]{images/grad_problem.png}
			\caption{ตัวอย่างภาพที่เกิดปัญหาเชิงตัวเลข}
			\label{image:rgb-space}
			\end{figure}
			\begin{align*}
			\tfrac{1}{| \nabla u |}=\tfrac{1}{\sqrt{u_x^2+u_y^2}} \rightarrow \infty
			\end{align*}
			\begin{align*}
			|\nabla u| \approx| \nabla u |_\beta=\sqrt{u_x^2+u_y^2+\beta},\ 0< \beta \ll 1
			\end{align*}
	\end{frame}
	\begin{frame}
		\frametitle{วิธีการสปริทเบรกแมน}
			\begin{align*}
		\min_{u,\boldsymbol{w}} \{ \mathcal{J}(u,\boldsymbol{w}) = \dfrac{1}{2} \int_{\Omega} \lambda(u-z)^2 d\Omega +  \int_{\Omega}  |\boldsymbol{w}|  d\Omega + \frac{\theta}{2} \int_{\Omega} (\boldsymbol{w} - \nabla u + \boldsymbol{b}) d\Omega \}
		\end{align*}
		$$ \Big \downarrow$$
		\begin{align*}
		u^{\text{New}}=\underset{u}{\arg\min} \{ \mathcal{J}_1(u) = \dfrac{1}{2} \int_{\Omega} \lambda(u-z)^2 d\Omega + \frac{\theta}{2} \int_{\Omega} (\boldsymbol{w}^{\text{old}} - \nabla u + \boldsymbol{b}^{\text{old}}) d\Omega \}
		\end{align*}
		\begin{align*}
		\boldsymbol{w}^{\text{New}}=\underset{\boldsymbol{w}}{\arg\min} \{ \mathcal{J}_2(\boldsymbol{w}) = \int_{\Omega}  | \boldsymbol{w}|  d\Omega  + \frac{\theta}{2} \int_{\Omega} (\boldsymbol{w} - \nabla u^{\text{New}} + \boldsymbol{b}^{\text{old}}) d\Omega \}
		\end{align*}
		\begin{align*}
		\boldsymbol{b}^{\text{New}}=\boldsymbol{b}^{\text{old}}+\nabla u^{\text{New}}-\boldsymbol{w}^{\text{New}}
		\end{align*}
	\end{frame}  
	\begin{frame}
		\frametitle{ประสิทธิภาพของวิธีการเชิงตัวเลขทั้ง 3 วิธี}
		\begin{table}[H]
		\centering
		\captionsetup{justification=centering}
			\begin{tabular}[ht]{|l|c|c|c|c|}
				\hline
				วิธีการ  & เวลาประมวล  (วินาที) & PSNR (dB) & SSIM \\
				\hline
				การเดินเวลา & 120.68 & 16.72 & 0.9960 \\
				การทำซ้ำจุดตรึง & 74.81 & 38.67 & 0.9999 \\
				การสปริทเบรกแมน & 14.06 & 39.42 & 0.9999  \\
				\hline
			\end{tabular}
		\caption{แสดงการซ่อมแซมเฉลี่ยของวิธีการเชิงตัวเลข \\ โดยที่ $\lambda = 250, \beta = 10^{-5}, \tau = 10^{-5}, \theta = 5 $}
		\end{table}	
	\end{frame}
	\begin{frame}
		\frametitle{ขั้นตอนวิธีที่พัฒนาขึ้น}

	\end{frame}
	\begin{frame}
		\frametitle{ขั้นตอนสำหรับการซ่อมแซมภาพศิลปะไทย}
	\end{frame}
	\begin{frame}
		\frametitle{คำตอบเริ่มต้น}
		\begin{figure}[H]
			\centering
			\begin{subfigure}{0.8\linewidth}
				\centering
				\includegraphics[width=1\linewidth]{images/image_inital_solution.png}
			\end{subfigure}
			\caption{วิธีการพีระมิดรูปภาพ}
		\end{figure}
	\end{frame}
	\begin{frame}
		\centering
		\Huge{ขอขอบคุณ}
	\end{frame}
\end{document}





