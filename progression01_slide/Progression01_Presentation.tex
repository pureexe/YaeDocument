\documentclass[xcolor=dvipsnames, xetex,serif]{beamer}
%\documentclass[handout,xetex,serif]{beamer} %ใช้บรรทัดนี้สำหรับปริ้นเอกสาร
\usepackage{color,amsmath,graphics,graphicx}
\usepackage{epsfig,amsfonts,graphics}
\usepackage{mathrsfs,hyperref}
\usepackage{subcaption,float,framed,algorithm2e,hyperref}
%===============================================
\usepackage{fontspec,xltxtra,xunicode}
\defaultfontfeatures{Scale=1.23}
\XeTeXlinebreaklocale “th_TH” % สำหรับตัดคำ
\setmainfont[Scale=1.23]{THSarabunNew}
% 1.23 เท่าคือจาก 12 pt บน LaTeX ให้เท่ากับ 16pt บน Word
%=====================================================
%\usepackage{pgfpages} %ใช้บรรทัดนี้สำหรับปริ้นเอกสาร
%\pgfpagesuselayout{4 on 1}[a4paper,border shrink=5mm,landscape]
%\pgfpagesuselayout{2 on 1}[a4paper,border shrink=5mm]
 %ใช้บรรทัดนี้สำหรับปริ้นเอกสาร

%%%%%%%%%%%%%%% THEOREM Environments %%%%%%%%%% 					
\newtheorem{conjecture}[theorem]{บทคาดการณ์}								
\newtheorem{remark}[theorem]{หมายเหตุ}										
\numberwithin{equation}{section}							
\renewcommand\tablename{ตารางที่}
\renewcommand\figurename{รูปที่}						
\renewcommand{\bibname}{บรรณานุกรม}						
\renewcommand{\indexname}{ดรรชนี}
\setbeamertemplate{caption}[numbered]	
\setbeamertemplate{theorems}[numbered]				
%%%%%%%%%%%%%%%%%%%%%%%%%%%%%%%%%%%%%%%%%%%%%%%

\mode<presentation>{
	\usetheme{Madrid}
	\usecolortheme[named=PineGreen]{structure}}
 \title[วิธีเชิงตัวเลขสำหรับต่อเติมภาพ]{\normalsize{ขั้นตอนวิธีเชิงตัวเลขชนิดใหม่สำหรับการต่อเติมภาพที่ใช้การแปรผันรวมกับการประยุกต์สำหรับซ่อมแซมภาพจิตรกรรมไทยโบราณและการลบบทบรรยายจากอนิเมะ\\A new numerical algorithm for TV-based image inpainting with its applications for restoring ancient Thai painting images and removing subtitles from animes}}
 \author[ภัคพล]{ภัคพล พงษ์ทวี}
 \institute[SU]{
 	ภาควิชาคณิตศาสตร์\\
 	มหาวิทยาลัยศิลปากร \\}
 \date[Project Progression]{การนำเสนอความก้าวหน้าโครงงานวิจัย\\
 	30 พฤศจิกายน 2561}
 
 \AtBeginSubsection[]{
 	\begin{frame}<beamer>
 		\frametitle{Outlines}
 		\tableofcontents [currentsection,currentsubsection]
 	\end{frame}}
 	%\setbeamertemplate{item}[square]
 	%============================================================================
 	\begin{document}
 		\begin{frame}
 			\titlepage 
 			 \note{แนะนำตัว, วันนี้จะมานำเสนอหัวข้อวิจัยเรื่อง (ชื่อเรื่อง)}
 		\end{frame}
 		%\begin{frame} \tableofcontents \end{frame}
 		%\begin{frame}\frametitle{การประมวลผลภาพ}\end{frame}
 		
 		\section{Introduction}		
		\begin{frame}
			\frametitle{ความก้าวหน้า}
			\begin{itemize}
				\item การซ่อมแซมภาพศิลปะไทย
				\item การลบคำบรรยายอนิเมะ
			\end{itemize}
		\end{frame}
		\begin{frame}
			\frametitle{ตัวแบบการต่อเติมภาพเฉดสีเทาที่ใช้การแปรผันรวม}
			\begin{align*}
			\min_{u} \{ \mathcal{J}(u) = \frac{1}{2} \int_{\Omega}\lambda (u-z)^2 d\Omega +  \int_{\Omega}  |\nabla u|  d\Omega \}
			\end{align*}
			 \vspace{1cm}
			\begin{align*}
			\lambda=\lambda(\mathbf{x}) = \left \{ \begin{array}{ll}  \lambda_0, & x \in \Omega \textbackslash D \\ 0, & x \in D  \end{array} \right . 
			\end{align*}
			\let\thefootnote\relax\footnotetext{\tiny{T.F. Chan and J. Shen , “Mathematical models of local non-texture inpaintings”, SIAM Journal on Applied Mathematics, vol. 62, no. 3, pp. 1019–1043, 2001.}}	
			%note{ในการต่อเติมภาพเฉดสีเทาคุณ Chan และ Shen  ได้นำเสนอตัวแบบเชิงการแปรผัน (variational model) ที่ใช้เร็กกิวลาร์ไรซ์เซชันแบบการแปรผันรวม โดยพัฒนาต่อจากตัวแบบ ROF ซึ่งใช้สำหรับการกำจัดสัญญาณรบกวน ซึ่งตัวแบบเชิงการแปรผันนี้กำหนดดังสมการนี้ และ lambda แทนพารามิเตอร์เร็กกิวลาร์ไรซ์เซชัน (regularization parameter) และ $\lambda_0 >0$}
		\end{frame} 
		\begin{frame}
			\frametitle{ตัวแบบการต่อเติมภาพเฉดสีเทาที่ใช้การแปรผันรวม (ต่อ)}
			\begin{align*}
			\min_{u} \{ \mathcal{J}(u) = \frac{1}{2} \int_{\Omega}\lambda (u-z)^2 d\Omega +  \int_{\Omega}  |\nabla u|  d\Omega \}
			\end{align*}
			$$ \Big \downarrow$$
			\begin{align*}
			\left \{ \begin{array}{ll}  - \nabla \cdot  \Big( \dfrac{\nabla u}{|\nabla u|} \Big) + \lambda (u-z) = 0,  & \hspace{1cm} \mathbf{x} \in (1,n)^2 \\ \dfrac{\partial u}{\partial \boldsymbol{n}} = 0, & \hspace{1cm} x \in \partial \Omega \end{array} \right .
			\end{align*}			
			%\note{โดยแคลคูลัสของการแปรผัน (Calculus of variations) จะได้สมการออยเลอร์ลากรางจ์ที่เกี่ยวข้องกัน ได้ดังนี้}
		\end{frame} 
		\begin{frame}
			\frametitle{การเดินเวลาแบบชัดแจ้ง (explicit time marching)}
			\begin{align*}
			u(\mathbf{x},t_{k+1})=u(\mathbf{x},t_{k})+\tau\left(\nabla \cdot\left(\dfrac{\nabla u (\mathbf{x},t_k)}{| \nabla u (\mathbf{x},t_k) | }\right) + \lambda(\mathbf{x})(u (\mathbf{x},t_k)-z(\mathbf{x})) \right)
			\end{align*}
			\begin{align*}
			u(\mathbf{x},t_0)=z \hspace{1cm} t_k=t_0+k\tau\ (\tau>0)  \hspace{1cm}  t_0=0
			\end{align*}
			\vspace{1cm}
			\begin{align*}
				u(\mathbf{x},t_0), u(\mathbf{x},t_1), u(\mathbf{x},t_2), u(\mathbf{x},t_3), ... ,  \textcolor{red}{u(\mathbf{x},t^{*})}
			\end{align*}
			\let\thefootnote\relax\footnotetext{\tiny{L. I. Rudin, S. Osher, E. Fatemi, “Nonlinear total variation based noise removal algorithms", Physica D: Nonlinear Phenomena, vol 60, issues 1–4, pp. 259-268, 1992.}}			
			%\note{ต่อไปจะกล่าวทบทวนวิธีการเชิงตัวเลขสำหรับแก้สมการเชิงอนุพันธ์ย่อย ซึ่งมีด้วยกันหลายวิธี  Rudin, Osher และ Fatemi ได้แนะนำวิธีการเชิงตัวเลขสำหรับการกำจัดสัญญาณรบกวนด้วยวิธีการเดินเวลาแบบชัดแจ้ง ซึ่งสามารถนำมาประยุกต์เพื่อใช้กับการต่อเติมภาพได้ เริ่มจากการแนะนําตัวแปรเวลาสังเคราะห์ จากนั้นหาคําตอบแบบสภาวะคงตัว เมื่อเวลาลู่เข้าสู่อนันต์}
		\end{frame} 
		\begin{frame}
			\frametitle{การทำซ้ำแบบจุดตรึง (fixed-point iteration)}
			\begin{align*}
				- \nabla\cdot\left(\dfrac{\nabla u^{[\nu+1]}}{{| \nabla u |}^{[v]} }\right) + \lambda(u^{[\nu+1]}-z)  = 0,\ u^{[0]}=z
			\end{align*}
			\vspace{1cm}
			\begin{align*}
			u^{[0]}, u^{[1]}, u^{[2]}, u^{[3]}, ..., \textcolor{red}{u^{*}}    
			\end{align*}
			\let\thefootnote\relax\footnotetext{\tiny{C.R. Vogel and M.E. Oman,“Iterative methods for total variation denoising", SIAM Journal on Scientific Computing. vol. 17, pp. 227-238, 1996.}}
			%\note{นอกจากนี้ยังมีคณะวิจัยของ Vogel และ Oman ได้แนะนำวิธีการทำซ้ำแบบจุดตรึงสำหรับการกำจัดสัญญาณรบกวนไว้ ซึ่งสามารถนำมาประยุกต์กับวิธีการต่อเติมภาพ เริ่มจากการแนะนำดัชนีการทำซ้ำแบบจุดตรึง u=0,1,2,... และนิยามรูปแบบการทำซ้ำโดย }
		\end{frame} 
		\begin{frame}
			\frametitle{ปัญหาเชิงตัวเลข}
				\begin{figure}[H]
				\centering
				\includegraphics[width=0.2\linewidth]{images/grayscale_inpaint/result_splitbergman.png}
				\caption{ตัวอย่างภาพที่เกิดปัญหาเชิงตัวเลข}
				\label{image:rgb-space}
				\end{figure}
				\begin{align*}
				\tfrac{1}{| \nabla u |}=\tfrac{1}{\sqrt{u_x^2+u_y^2}} \rightarrow \infty
				\end{align*}
				\begin{align*}
				|\nabla u| \approx| \nabla u |_\beta=\sqrt{u_x^2+u_y^2+\beta},\ 0< \beta \ll 1
				\end{align*}
				% \note{แต่ในบริเวณที่ u มีความเข้มสีเป็นเอกพันธ์ุ จะทำให้ (ชี้บรรทัดบน) 1ส่วนขนาดของแกรด u ลู่ออกสู่อนันต์ เพื่อหลีกเลี่ยงปัญหาเชิงตัวเลขจะเกิดขึ้นใน เราจะใช้ (ชี้บรรทัดล่าง) การประมาณขนาดของแกรดโดยเพิ่ม beta เข้าไป โดยค่า beta นี้ มีค่าน้อยลงมากขึ้นเท่าไหร่ ความแม่นยำของตัวแบบยิ่งมีมากขึ้นเท่านั้น แต่ เรายังพบอีกว่าการแก้สมการยิ่งมีความยุ่งยากมากขึ้นสำหรับ beta น้อยๆ }
		\end{frame}
		\begin{frame}
			\frametitle{วิธีการสปริทเบรกแมน (Split Bregman method)}
			\begin{align*}
			\min_{u,\boldsymbol{w}} \{ \mathcal{J}(u,\boldsymbol{w}) = \dfrac{1}{2} \int_{\Omega} \lambda(u-z)^2 d\Omega +  \int_{\Omega}  |\nabla \boldsymbol{w}|  d\Omega + \frac{\theta}{2} \int_{\Omega} (\boldsymbol{w} - \nabla u + \boldsymbol{b}) d\Omega \}
			\end{align*}
			\let\thefootnote\relax\footnotetext{\tiny{T. Goldstein and S. Osher,“The Split Bregman Method for L1-Regularized Problems", SIAM Journal on Imaging Sciences. vol. 2, issue 2, pp. 323-343, 2009.}}
			%\note{เพื่อเอาชนะความยากเชิงตัวเลขนี้ คณะวิจัยโดยคุณ Goldstein และ Osher ได้แนะนำวิธีการสปริทเบรกแมนซึ่งสามารถกล่าวถึงพอสังเขป โดยการใช้พารามิเตอร์เสริม w ในตัวแบบเชิงการแปรผัน และเพิ่ม b ซึ่งเป็นตัวแปร bergman}
		\end{frame} 
		\begin{frame}
			\frametitle{วิธีการสปริทเบรกแมน (ต่อ)}
				\begin{align*}
			\min_{u,\boldsymbol{w}} \{ \mathcal{J}(u,\boldsymbol{w}) = \dfrac{1}{2} \int_{\Omega} \lambda(u-z)^2 d\Omega +  \int_{\Omega}  |\nabla \boldsymbol{w}|  d\Omega + \frac{\theta}{2} \int_{\Omega} (\boldsymbol{w} - \nabla u + \boldsymbol{b}) d\Omega \}
			\end{align*}
			$$ \Big \downarrow$$
			\begin{align*}
			u^{\text{New}}=\underset{u}{\arg\min} \{ \mathcal{J}_1(u) = \dfrac{1}{2} \int_{\Omega} \lambda(u-z)^2 d\Omega + \frac{\theta}{2} \int_{\Omega} (\boldsymbol{w}^{\text{old}} - \nabla u + \boldsymbol{b}^{\text{old}}) d\Omega \}
			\end{align*}
			\begin{align*}
			\boldsymbol{w}^{\text{New}}=\underset{\boldsymbol{w}}{\arg\min} \{ \mathcal{J}_2(\boldsymbol{w}) = \int_{\Omega}  |\nabla \boldsymbol{w}|  d\Omega  + \frac{\theta}{2} \int_{\Omega} (\boldsymbol{w} - \nabla u^{\text{New}} + \boldsymbol{b}^{\text{old}}) d\Omega \}
			\end{align*}
			\begin{align*}
			\boldsymbol{b}^{\text{New}}=\boldsymbol{b}^{\text{old}}+\nabla u^{\text{New}}-\boldsymbol{w}^{\text{New}}
			\end{align*}
			%\note{เราจะใช้วิธีการหาค่าต่ำที่สุดแบบสลับ โดยทำการตรึงค่า w จากนั้นเริ่มแก้ปัญหาย่อย แล้วนำ u ที่ได้มาตรึงค่า u เพื่อแก้ปัญหาย่อย w จากนั้นจึงทำการเปลี่ยนค่าตัวแปร b แล้วเริ่มทำซ้ำที่การหา u ใหม่อีกครั้ง จนกระทั่งค่านอร์มระหว่างรูป u ปัจจุบันและก่อนหน้าน้อยกว่าค่าที่กำหนด}
		\end{frame}  
		\begin{frame}
			\frametitle{Peak Signal Noise Ratio (PSNR)}
			\begin{align*}
			\text{PSNR}  = 10 \cdot log_{10} ( \frac{1}{\sqrt{\text{MSE}}} )
			\end{align*}
			\begin{itemize}
				\item[$\bullet$] MSE คือค่าคลาดเคลื่อนกำลังสองเฉลี่ยของภาพ โดยที่ MSE = $\frac{1}{nx \times ny} \sum (u - \tilde{u})^2 $
				\item[$\bullet$] $u$ แทนภาพต้นฉบับ
				\item[$\bullet$] $\tilde{u}$  แทนภาพที่ได้จากการซ่อมแซมโดยวิธีเชิงตัวเลข
			\end{itemize}
		\end{frame}
		\begin{frame}
			\frametitle{Structural Similarity (SSIM)}
			\begin{align*}
			\text{SSIM}(u,\tilde{u}) = \frac{(2\mu_u\mu_{\tilde{u}} + 0.0001)(2\sigma_{u\tilde{u}} + 0.0009)}{(\mu_u^2+\mu_{\tilde{u}}^2+0.0001)(\sigma_u^2+\sigma_{\tilde{u}}^2+0.0009)}
			\end{align*}
			\begin{itemize}
				\item[$\bullet$] $u$ แทนภาพต้นฉบับ
				\item[$\bullet$] $\tilde{u}$  แทนภาพต้นฉบับ และภาพที่ได้จากการซ่อมแซมโดยวิธีเชิงตัวเลข
				\item[$\bullet$] $\mu_u$ คือค่าเฉลี่ยของ $u$
				\item[$\bullet$] $\mu_{\tilde{u}}$ คือค่าเฉลี่ยของ $\tilde{u}$
				\item[$\bullet$]  $\sigma_u$ คือความแปรปรวนของ $u$ 
				\item[$\bullet$] $\sigma_{\tilde{u}}$ คือความแปรปรวนของ $\tilde{u}$
			\end{itemize}
		\end{frame}
		\begin{frame}
			\frametitle{ภาพสังเคราะห์}
			\begin{figure}[H]
				\centering
				\begin{subfigure}{0.3\linewidth}
					\centering
					\includegraphics[width=0.7\linewidth]{images/image_inpaint_synthetic/case01-original.png}
				\end{subfigure}
				\begin{subfigure}{0.3\linewidth}
					\centering
					\includegraphics[width=0.7\linewidth]{images/image_inpaint_synthetic/case02-original.png}
				\end{subfigure}
				\begin{subfigure}{0.3\linewidth}
					\centering
					\includegraphics[width=0.7\linewidth]{images/image_inpaint_synthetic/case03-original.png}
				\end{subfigure}
				\begin{subfigure}{0.3\linewidth}
					\centering
					\includegraphics[width=0.7\linewidth]{images/image_inpaint_synthetic/case04-original.png}
				\end{subfigure}
				\begin{subfigure}{0.3\linewidth}
					\centering
					\includegraphics[width=0.7\linewidth]{images/image_inpaint_synthetic/case05-original.png}	
				\end{subfigure}
				\caption{ภาพต้นฉบับ}
				\label{image:synart_original}
			\end{figure}			
		\end{frame}
		\begin{frame}
			\frametitle{ภาพสังเคราะห์ที่จะซ่อมแซม}
			\begin{figure}[H]
				\centering
				\begin{subfigure}{0.3\linewidth}
					\centering
					\includegraphics[width=0.7\linewidth]{images/image_inpaint_synthetic/case01-toinpaint.png}
				\end{subfigure}
				\begin{subfigure}{0.3\linewidth}
					\centering
					\includegraphics[width=0.7\linewidth]{images/image_inpaint_synthetic/case02-toinpaint.png}
				\end{subfigure}
				\begin{subfigure}{0.3\linewidth}
					\centering
					\includegraphics[width=0.7\linewidth]{images/image_inpaint_synthetic/case03-toinpaint.png}
				\end{subfigure}
				\bigskip
				\begin{subfigure}{0.3\linewidth}
					\centering
					\includegraphics[width=0.7\linewidth]{images/image_inpaint_synthetic/case04-toinpaint.png}
				\end{subfigure}
				\begin{subfigure}{0.3\linewidth}
					\centering
					\includegraphics[width=0.7\linewidth]{images/image_inpaint_synthetic/case05-toinpaint.png}
				\end{subfigure}
				\caption{ภาพที่จะทำการซ่อมแซม}
			\end{figure}
		\end{frame}
		\begin{frame}
			\frametitle{ผลการทดสอบ}
			\begin{figure}[H]
				\centering
				\begin{subfigure}{0.15\linewidth}
					\centering
					\includegraphics[width=0.8\linewidth]{images/result_ex1/timemarch01.png}
				\end{subfigure}
				\begin{subfigure}{0.15\linewidth}
					\centering
					\includegraphics[width=0.8\linewidth]{images/result_ex1/timemarch02.png}
				\end{subfigure}
				\begin{subfigure}{0.15\linewidth}
					\centering
					\includegraphics[width=0.8\linewidth]{images/result_ex1/timemarch03.png}			
				\end{subfigure}
				\begin{subfigure}{0.15\linewidth}
					\centering
					\includegraphics[width=0.8\linewidth]{images/result_ex1/timemarch04.png}			
				\end{subfigure}
				\begin{subfigure}{0.15\linewidth}
					\centering
					\includegraphics[width=0.8\linewidth]{images/result_ex1/timemarch05.png}			
				\end{subfigure}
				\caption{ผลการซ่อมแซมจากวิธีการเดินเวลา}
				\label{result:image-timemarching}
			\end{figure}
			\begin{figure}[H]
				\centering
				\begin{subfigure}{0.15\linewidth}
					\centering
					\includegraphics[width=0.8\linewidth]{images/result_ex1/fixpoint01.png}
				\end{subfigure}
				\begin{subfigure}{0.15\linewidth}
					\centering
					\includegraphics[width=0.8\linewidth]{images/result_ex1/fixpoint02.png}
				\end{subfigure}
				\begin{subfigure}{0.15\linewidth}
					\centering
					\includegraphics[width=0.8\linewidth]{images/result_ex1/fixpoint03.png}			
				\end{subfigure}
				\begin{subfigure}{0.15\linewidth}
					\centering
					\includegraphics[width=0.8\linewidth]{images/result_ex1/fixpoint04.png}			
				\end{subfigure}
				\begin{subfigure}{0.15\linewidth}
					\centering
					\includegraphics[width=0.8\linewidth]{images/result_ex1/fixpoint05.png}			
				\end{subfigure}
				\caption{ผลการซ่อมแซมจากวิธีการทำซ้ำแบบจุดตรึง}
			\end{figure}
			\begin{figure}[H]
				\centering
				\begin{subfigure}{0.15\linewidth}
					\centering
					\includegraphics[width=0.8\linewidth]{images/result_ex1/splitbergman01.png}
				\end{subfigure}
				\begin{subfigure}{0.15\linewidth}
					\centering
					\includegraphics[width=0.8\linewidth]{images/result_ex1/splitbergman02.png}
				\end{subfigure}
				\begin{subfigure}{0.15\linewidth}
					\centering
					\includegraphics[width=0.8\linewidth]{images/result_ex1/splitbergman03.png}			
				\end{subfigure}
				\begin{subfigure}{0.15\linewidth}
					\centering
					\includegraphics[width=0.8\linewidth]{images/result_ex1/splitbergman04.png}			
				\end{subfigure}
				\begin{subfigure}{0.15\linewidth}
					\centering
					\includegraphics[width=0.8\linewidth]{images/result_ex1/splitbergman05.png}			
				\end{subfigure}
				\caption{ผลการซ่อมแซมจากวิธีการสปริทเบรกแมน}
				\label{result:image-splitbregman}
			\end{figure}
		\end{frame} 
		\begin{frame}
			\frametitle{ประสิทธิภาพของวิธีการเชิงตัวเลขทั้ง 3 วิธี}
			\begin{table}[H]
			\centering
				\begin{tabular}[ht]{|l|c|c|c|c|}
					\hline
					วิธีการ  & เวลาประมวล  (วินาที) & PSNR (dB) & SSIM \\
					\hline
					การเดินเวลา & 120.68 & 16.72 & 0.9960 \\
					การทำซ้ำจุดตรึง & 74.81 & 38.67 & 0.9999 \\
					การสปริทเบรกแมน & 14.06 & 39.42 & 0.9999  \\
					\hline
				\end{tabular}
			\caption{แสดงการซ่อมแซมเฉลี่ยของวิธีการเชิงตัวเลข}
			\end{table}	
		\end{frame}
		\begin{frame}
			\frametitle{พีระมิดรูปภาพ}
			\begin{figure}[H]
				\centering
				\begin{subfigure}{0.6\linewidth}
					\centering
					\includegraphics[width=0.8\linewidth]{images/image_inpaint_synthetic/image_pyramid.png}
				\end{subfigure}
				\caption{วิธีการพีระมิดรูปภาพ}
			\end{figure}
		\end{frame}
		\begin{frame}
			\frametitle{พีระมิดรูปภาพ (ต่อ)}
			\begin{figure}[H]
				\centering
				\begin{subfigure}{0.8\linewidth}
					\centering
					\includegraphics[width=1\linewidth]{images/image_inpaint_synthetic/image_inital_solution.png}
				\end{subfigure}
				\caption{วิธีการพีระมิดรูปภาพ}
			\end{figure}
		\end{frame}
		\begin{frame}
			\frametitle{ผลการซ่อมแซมเมื่อใช้พีระมิดรูปภาพ}
			\begin{table}[H]
				\centering
				\begin{tabular}[ht]{|l|c|c|c|c|}
					\hline
					รูปแบบการทำซ้ำ  & เวลาประมวล  (วินาที) & PSNR (dB) & SSIM \\
					\hline
					ไม่ใช้พีระมิดรูปภาพ & 17.38 & 39.42 & 0.9999 \\
					10/1/1/10000 & 13.52 & 39.38 & 0.9999 \\
					10/3/3/10000 & 11.86 & 39.54 & 0.9999 \\
					10/10/10/10000 & 9.26 & 40.17 & 0.9999\\
					100/1/1/10000 & 10.28 & 39.04 & 0.9999\\
					100/3/3/10000 & 10.28 & 39.80 & 0.9999\\
					100/10/10/10000 & 9.27 & 40.12 & 0.9999 \\
					\hline
				\end{tabular}
				\caption{ผลการซ่อมแซมภาพโดยวิธีการเชิงตัวเลขที่นำเสนอในรูปของค่าเฉลี่ย}
				\label{result:table-multiresolution1-summary}
			\end{table}	
		\end{frame}
		\begin{frame}
			\frametitle{การทำซ้ำความละเอียดคมชัดสุด}
			\begin{figure}[H]
				\centering
				\begin{subfigure}{0.4\linewidth}
					\centering
					\includegraphics[width=0.6\linewidth]{images/just10enough/only5time.png}
					\caption{5 ครั้ง}
				\end{subfigure}
				\begin{subfigure}{0.4\linewidth}
					\centering
					\includegraphics[width=0.6\linewidth]{images/just10enough/only10time.png}
					\caption{10 ครั้ง}
				\end{subfigure}
				\begin{subfigure}{0.4\linewidth}
					\centering
					\includegraphics[width=0.6\linewidth]{images/just10enough/only50time.png}			
					\caption{50 ครั้ง}
				\end{subfigure}
				\begin{subfigure}{0.4\linewidth}
					\centering
					\includegraphics[width=0.6\linewidth]{images/just10enough/only100time.png}			
					\caption{100 ครั้ง}
				\end{subfigure}
				\caption{{\footnotesize พีระมิดที่ลำดับการทำซ้ำเป็น 10/10/10 และที่ระดับความคมชัดละเอียดสุดใช้จำนวนการทำซ้ำที่ต่างกัน}}
			\end{figure}
		\end{frame}
		\begin{frame}
			\frametitle{ผลการซ่อมแซมภาพในระดับความคมชัดละเอียดสุด 10 ครั้ง}
			\begin{table}[H]
				\centering
				\begin{tabular}[ht]{|l|c|c|c|c|}
					\hline
					รูปแบบการทำซ้ำ  & เวลาประมวล  (วินาที) & PSNR (dB) & SSIM \\
					\hline
					ไม่ใช้พีระมิดรูปภาพ & 0.37 & 17.26 & 0.9963  \\
					10/1/1/10 & 0.40 & 28.54 & 0.9993 \\
					10/3/3/10 & 0.33 & 29.83  & 0.9994 \\
					10/10/10/10 & 0.38 & 32.56 & 0.9995 \\
					100/1/1/10 & 0.34 & 31.50 & 0.9999 \\
					100/3/3/10 & 0.36 & 31.99 & 0.9999 \\
					100/10/10/10 & 0.38 & 33.39 & 0.9998 \\
					\hline
				\end{tabular}
				\caption{{\small ผลการซ่อมแซมภาพเฉลี่ยเมื่อใช้การทำซ้ำในระดับความคมชัดละเอียดสุด 10 ครั้ง}}
			\end{table}	
		\end{frame}
		\begin{frame}
			\frametitle{การซอมแซมภาพจิตรกรรมไทยโบราณ}
			\begin{figure}[H]
				\centering
				\begin{subfigure}{0.3\linewidth}
					\centering
					\includegraphics[width=0.7\linewidth]{images/thaiart/case01-original.png}
					\caption{วัดแก้วไพฑูรย์}
					\label{image:thaiart_case01_original}
				\end{subfigure}
				\begin{subfigure}{0.3\linewidth}
					\centering
					\includegraphics[width=0.7\linewidth]{images/thaiart/case02-original.png}
					\caption{วัดแก้วไพฑูรย์}
					\label{image:thaiart_case02_original}
				\end{subfigure}
				\begin{subfigure}{0.3\linewidth}
					\centering
					\includegraphics[width=0.7\linewidth]{images/thaiart/case03-original.png}
					\caption{วัดพระยืนพุทธบาทยุคล}
					\label{image:thaiart_case03_original}			
				\end{subfigure}		
				\begin{subfigure}{0.3\linewidth}
					\centering
					\includegraphics[width=0.7\linewidth]{images/thaiart/case04-original.png}
					\caption{วัดคงคาราม}
					\label{image:thaiart_case04_original}			
				\end{subfigure}
				\begin{subfigure}{0.3\linewidth}
					\centering
					\includegraphics[width=0.7\linewidth]{images/thaiart/case05-original.png}
					\caption{วัดท่าถนน}
					\label{image:thaiart_case05_original}			
				\end{subfigure}
				\begin{subfigure}{0.3\linewidth}
					\centering
					\includegraphics[width=0.7\linewidth]{images/thaiart/inpaint-domain.png}
					\caption{รอยความเสียหาย}
				\end{subfigure}
				\caption{ภาพต้นฉบับสำหรับใช้ในการทดสอบ}
			\end{figure}
		\end{frame}
		\begin{frame}
			\frametitle{ภาพที่ทำให้เสียหาย}
			\begin{figure}[H]
				\centering
				\begin{subfigure}{0.3\linewidth}
					\centering
					\includegraphics[width=0.7\linewidth]{images/thaiart/case01-toinpaint.png}
				\end{subfigure}
				\begin{subfigure}{0.3\linewidth}
					\centering
					\includegraphics[width=0.7\linewidth]{images/thaiart/case02-toinpaint.png}
				\end{subfigure}
				\vspace{1cm}
				\begin{subfigure}{0.3\linewidth}
					\centering
					\includegraphics[width=0.7\linewidth]{images/thaiart/case03-toinpaint.png}			
				\end{subfigure}
				\begin{subfigure}{0.3\linewidth}
					\centering
					\includegraphics[width=0.7\linewidth]{images/thaiart/case04-toinpaint.png}			
				\end{subfigure}
				\begin{subfigure}{0.3\linewidth}
					\centering
					\includegraphics[width=0.7\linewidth]{images/thaiart/case05-toinpaint.png}			
				\end{subfigure}
				\caption{ภาพที่ทำให้เสียหาย}
			\end{figure}
		\end{frame}
		\begin{frame}
			\frametitle{ผลการซ่อมแซมโดยวิธีการสปริทเบรกแมน}
			\begin{figure}[H]
				\centering
				\begin{subfigure}{0.3\linewidth}
					\centering
					\includegraphics[width=0.7\linewidth]{images/result_ex4/splitbergman_case01.png}
				\end{subfigure}
				\begin{subfigure}{0.3\linewidth}
					\centering
					\includegraphics[width=0.7\linewidth]{images/result_ex4/splitbergman_case02.png}
				\end{subfigure}
				\vspace{1cm}
				\begin{subfigure}{0.3\linewidth}
					\centering
					\includegraphics[width=0.7\linewidth]{images/result_ex4/splitbergman_case03.png}			
				\end{subfigure}
				\begin{subfigure}{0.3\linewidth}
					\centering
					\includegraphics[width=0.7\linewidth]{images/result_ex4/splitbergman_case04.png}			
				\end{subfigure}
				\begin{subfigure}{0.3\linewidth}
					\centering
					\includegraphics[width=0.7\linewidth]{images/result_ex4/splitbergman_case05.png}			
				\end{subfigure}
				\caption{ผลการซ่อมแซมโดยวิธีการสปริทเบรกแมน}
			\end{figure}
		\end{frame}
		\begin{frame}
			\frametitle{ผลการซ่อมแซมโดยวิธีการที่พัฒนาขึ้น}
			\begin{figure}[H]
				\centering
				\begin{subfigure}{0.3\linewidth}
					\centering
					\includegraphics[width=0.7\linewidth]{images/result_ex4/multisplitbergman_case01.png}
				\end{subfigure}
				\begin{subfigure}{0.3\linewidth}
					\centering
					\includegraphics[width=0.7\linewidth]{images/result_ex4/multisplitbergman_case02.png}
				\end{subfigure}
				\vspace{1cm}
				\begin{subfigure}{0.3\linewidth}
					\centering
					\includegraphics[width=0.7\linewidth]{images/result_ex4/multisplitbergman_case03.png}			
				\end{subfigure}
				\begin{subfigure}{0.3\linewidth}
					\centering
					\includegraphics[width=0.7\linewidth]{images/result_ex4/multisplitbergman_case04.png}			
				\end{subfigure}
				\begin{subfigure}{0.3\linewidth}
					\centering
					\includegraphics[width=0.7\linewidth]{images/result_ex4/multisplitbergman_case05.png}			
				\end{subfigure}
				\caption{ผลการซ่อมแซมโดยวิธีการที่พัฒนาขึ้น}
			\end{figure}
		\end{frame}
		\begin{frame}
			\frametitle{เปรียบเทียบระหว่าง 2 วิธี}
		\end{frame}
		\begin{frame}
			\frametitle{ผลการซ่อมแซมภาพศิลปะไทย}
			\begin{table}[H]
				\centering
				\begin{tabular}[ht]{|l|c|c|c|c|}
					\hline
					วิธีการ  & เวลาประมวล  (วินาที) & PSNR (dB) & SSIM \\
					\hline
					สปริทเบรกแมน & 2.72 & 34.89 & 1.0000 \\ 
					วิธีการที่พัฒนาขึ้น & 0.39 & 35.30 & 1.0000 \\
					\hline
				\end{tabular}
				\caption{แสดงผลการซ่อมแซมภาพศิลปะไทยในรูปค่าเฉลี่ย}
				\label{result:table-thaiart-summary}
			\end{table}	
		\end{frame}
		\begin{frame}
			\frametitle{การลบบทบรรยายบนอนิเมะ}
			\begin{figure}[H]
				\centering
				\includegraphics[width=0.8\linewidth]{images/inori-preview.png}
				\caption{ Festival Asia Special Video - feat. Inori Aizawa }
			\end{figure}
		\end{frame}
		\begin{frame}
			\frametitle{Lorem Ipsum}
				\begin{figure}[H]
				\centering
				\includegraphics[width=0.85\linewidth]{images/lorem4lang.png}
				\caption{Lorem Ipsum ทั้ง 4 ภาษา \footnote{ \tiny{ส่วนหนึ่งของ Lorem ipsum  ในภาษาต่างๆ จาก https://en.wikipedia.org/wiki/Lorem\_ipsum,  https://ja.wikipedia.org/wiki/Lorem\_ipsum และ  https://th.wikipedia.org/wiki/ลอเร็มอิปซัม สืบค้นเมื่อ 25 พฤศจิกายน 2561} } }
			\end{figure}
		\end{frame}
		\begin{frame}
			\frametitle{แบ่งเป็น 5 ส่วน}
				\begin{figure}[H]
				\centering
				\includegraphics[width=0.8\linewidth]{images/inori-subbed-preview.png}
				\caption{แบ่งวิดีโอออกเป็น 5 ส่วน}
			\end{figure}
		\end{frame}
		\begin{frame}
			\frametitle{การหาบทบรรยาย}
			\begin{figure}[H]
				\begin{subfigure}{0.4\linewidth}
					\centering
					\includegraphics[width=0.8\linewidth]{images/subtitle_detection/detection-original.png}
					\caption{ภาพเฟรมอนิเมะบริเวณที่มีบทบรรยาย}
				\end{subfigure}
				\begin{subfigure}{0.4\linewidth}
					\centering
					\includegraphics[width=0.8\linewidth]{images/subtitle_detection/detection-threshold.png}
					\caption{ให้สีดำในภาพเป็นสีขาว สีอื่นเป็นสีดำ }
				\end{subfigure}
				\begin{subfigure}{0.4\linewidth}
					\centering
					\includegraphics[width=0.8\linewidth]{images/subtitle_detection/detection-inverse.png}
					\caption{สลับสีขาวเป็นสีดำ สีดำเป็นสีขาว}
				\end{subfigure}
				\begin{subfigure}{0.4\linewidth}
					\centering
					\includegraphics[width=0.8\linewidth]{images/subtitle_detection/detection-blackfill.png}
					\caption{ภาพหลังทำการเปลี่ยนพื้นที่สีขาว}
				\end{subfigure}
				\begin{subfigure}{0.4\linewidth}
					\centering
					\includegraphics[width=0.8\linewidth]{images/subtitle_detection/detection-erode-opening.png}
					\caption{ลบสิ่งที่เล็กเกินไปหรือใหญ่เกินไป}
				\end{subfigure}
				\begin{subfigure}{0.4\linewidth}
					\centering
					\includegraphics[width=0.8\linewidth]{images/subtitle_detection/detection-stoke.png}
					\caption{ขยายโดเมนต่อเติมให้ครอบคลุม}
				\end{subfigure}
				\caption{ขั้นตอนวิธีการหาบทบรรยาย}
			\end{figure}
		\end{frame}
		\begin{frame}
			\frametitle{ผลลัพธ์การหาบทบรรยาย}
			\begin{table}[H]
				\centering
				\footnotesize
				\begin{tabular}[ht]{|l|c|c|c|c|}
					\hline
					ภาษา  & จำนวนพิกเซลในโดเมน & จำนวนพิกเซลที่ตรวจพบ & จำนวนพิกเซลที่ผิดพลาด & ร้อยละการผิดพลาด \\
					\hline
					ไทย & 23,222,220 & 24,083,125 & 2,141,201 & 9.22 \\
					อังกฤษ & 27,278,745 & 28,598,424 & 3,714,321 & 13.62 \\
					ญี่ปุ่น & 28,544,173 & 30,103,466 & 3,740,971 & 13.11 \\
					\hline
				\end{tabular}
				\caption{ความคลาดเคลื่อนเฉลี่ยของการหาโดเมนต่อเติม ในบทบรรยายภาษาต่างๆ}
			\end{table}	
		\end{frame}
		\begin{frame}
			\frametitle{การข้ามเฟรมและยืมเฟรม}
			\begin{figure}[H]
				\centering
				\begin{subfigure}{0.4\linewidth}
					\centering
					\includegraphics[width=0.95\linewidth]{images/skipborrow/prevframe.png}
					\caption{เฟรมก่อนหน้า}
					\label{image:ssim_location_prev}
				\end{subfigure}
				\begin{subfigure}{0.4\linewidth}
					\centering
					\includegraphics[width=0.95\linewidth]{images/skipborrow/currentframe.png}
					\caption{เฟรมปัจจุบัน}
					\label{image:ssim_location_curr}
				\end{subfigure}
				\bigskip
				\begin{subfigure}{0.4\linewidth}
					\centering
					\includegraphics[width=0.95\linewidth]{images/skipborrow/prevframeinverse.png}
					\caption{บริเวณคำนวณ SSIM ของเฟรมก่อนหน้า}
					\label{image:ssim_location_prev_inv}
				\end{subfigure}
				\begin{subfigure}{0.4\linewidth}
					\centering
					\includegraphics[width=0.95\linewidth]{images/skipborrow/currentframeinverse.png}
					\caption{บริเวณคำนวณ SSIM ของเฟรมปัจจุบัน}
					\label{image:ssim_location_curr_inv}
				\end{subfigure}
				\caption{บริเวณที่คำนวณ SSIM สำหรับการยืมเฟรมและข้ามเฟรม}
				\label{image:ssim_location}
			\end{figure}
		\end{frame}
		\begin{frame}
			\frametitle{ผลการลบโดยใช้วิธีข้ามเฟรมและยืมเฟรม}
			\begin{table}[H]
				\centering
				\begin{tabular}[ht]{|l|c|c|c|c|}
					\hline
					วิธีการ  & เวลาประมวล  (วินาที) & PSNR (dB) & SSIM \\
					\hline
					สปริทเบรกแมนและพีระมิดรูปภาพ & 141.29 & 31.39  &  0.9510\\
					ยืมเฟรม & 132.78 & 32.20 & 0.9655\\
					ข้ามเฟรม & 89.29 & 29.07 & 0.9408 \\
					ยืมเฟรมและข้ามเฟรม & 75.76 & 29.33 & 0.9454 \\
					\hline
				\end{tabular}
				\caption{การลบคำบรรยายเฉลี่ยด้วยวิธีต่างๆ}
			\end{table}	
		\end{frame}
		\begin{frame}
			\frametitle{ผลการลบคำบรรยาย}
			\begin{figure}[H]
				\centering
				\begin{subfigure}{0.4\linewidth}
					\centering
					\includegraphics[width=0.80\linewidth]{images/subtitle-remove/beforesubtitleremove.png}
					\caption{ก่อนลบคำบรรยาย}
					\label{image:ssim_location_prev}
				\end{subfigure}
				\begin{subfigure}{0.4\linewidth}
					\centering
					\includegraphics[width=0.80\linewidth]{images/subtitle-remove/aftersubtitleremove.png}
					\caption{หลังลบคำบรรยาย}
					\label{image:ssim_location_curr}
				\end{subfigure}
				\caption{ก่อนลบคำบรรยายและหลังลบคำบรรยาย}
			\end{figure}
		\end{frame}
		\begin{frame}
			\frametitle{ผลการลบคำบรรยาย (ต่อ)}
			\begin{table}[H]
				\centering
				\begin{tabular}[ht]{|l|c|c|c|c|}
					\hline
					วิธีการ  & เวลาประมวล  (วินาที) & PSNR (dB) & SSIM \\
					\hline
					สปริทเบรกแมน & * & * & * \\
					วิธีการที่พัฒนาขึ้น & 75.76 & 29.33 & 0.9454 \\
					\hline
				\end{tabular}
				\caption{ผลการลบบทบรรยายออกจากอนิเมะโดยวิธีการสปริทเบรกแมนและวิธีการที่พัฒนาขึ้น}
			\end{table}	
		\end{frame}
		\begin{frame}
			\frametitle{วิธีการดำเนินงาน}
			\scalebox{0.7}{\begin{tabular}[ht]{|l|c|c|c|c|c|c|c|c|c|c|c|c|}
					\hline
					&\multicolumn{12}{c|}{เดือนที่}\\
					\cline{2-13}
					แผนการดำเนินงาน&1&2&3&4&5&6&7&8&9&10&11&12\\
					\hline
					ศึกษาตัวแบบและขั้นตอนวิธีการต่อเติมภาพที่ใช้การแปรผันรวมในเชิงลึก&x&x& & & & & & & & & &\\
					\hline
					พัฒนาขั้นตอนวิธีสำหรับการต่อเติมภาพที่ใช้การแปรผันรวมชนิดใหม่& & &x&x&x&x& & & & & &\\
					\hline
					ทดสอบขั้นตอนวิธีการต่อเติมภาพที่พัฒนาขึ้นโดยโปรแกรม- & & & &x &x&x& & & & & &\\
					คอมพิวเตอร์บนภาพสังเคราะห์และภาพจริง & & & & & & & & & & & &\\
					\hline
					อภิปรายผลที่ได้จากการทดลองเชิงตัวเลข & & & & & &x&x&x& & & &\\
					\hline
					สรุปผลการดำเนินงานวิจัยและจัดทำรูปเล่มฉบับสมบูรณ์& & & & & & & & &x&x&x&x\\
					\hline
			\end{tabular}}
		\end{frame}
		\begin{frame}
			\centering
			\Huge{ขอขอบคุณ}
		\end{frame}
	\end{document}





